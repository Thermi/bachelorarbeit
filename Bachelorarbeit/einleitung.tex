% This file is part of Bachelorarbeit

% Bachelorarbeit is free software: you can redistribute it and/or modify
% it under the terms of the GNU General Public License version 3 as published by
% the Free Software Foundation.

% Bachelorarbeit is distributed in the hope that it will be useful,
% but WITHOUT ANY WARRANTY; without even the implied warranty of
% MERCHANTABILITY or FITNESS FOR A PARTICULAR PURPOSE.  See the
% GNU General Public License for more details.

% You should have received a copy of the GNU General Public License
% along with Foobar. If not, see <http://www.gnu.org/licenses/>.
\section{Einleitung}
\subsection{Abstract}
Der Mangel an Unterstützung für TUN/TAP-Geräte unter Windows von strongSwan, sowie Fehler und
fehlende Unterstützung für Mehrfaktorauthentifizierung im nativen IPsec-Stack von modernen
Windows-Versionen zusammen mit dem Mangel an frei verfügbaren Alternativen zu den mangelhaften existierenden Lösungen
lässt den Bedarf an einem Client für IPsec-Roadwarrior-VPNs ungedeckt.

Diese Arbeit zielt darauf ab, Unterstützung für den OpenVPN TAP-Treiber
für Windows in strongSwan zu implementieren, sodass die Grundlage für die Implementierung
eines offenen Clients auf Basis von strongSwan gelegt ist.

\subsection{Motivation}
Die Motivation für die Arbeit ist der Mangel eines modernen, flexiblen, offenen
und  gepflegten IPsec-VPN-Clients für die Windows-Platform.
Aufgrund der Erfahrungen aus der Unterstützung des strongSwan-Projekts wurde
der Bedarf für so eine Software entdeckt, der durch die Ergänzung von fehlender
Funktionalität abgedeckt wird.
Der Mangel an starker Kryptografie und existierende Implementierungsfehler im nativen Client
von neueren Windowsversionen wird oft bemängelt und ist im Jahr 2016 nicht mehr adequat
zur Bedrohungslage durch staatliche Angreifer.

\BlankPage
\phantomsection

