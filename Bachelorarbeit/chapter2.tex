% This file is part of Bachelorarbeit

% Bachelorarbeit is free software: you can redistribute it and/or modify
% it under the terms of the GNU General Public License version 3 as published by
% the Free Software Foundation.

% Bachelorarbeit is distributed in the hope that it will be useful,
% but WITHOUT ANY WARRANTY; without even the implied warranty of
% MERCHANTABILITY or FITNESS FOR A PARTICULAR PURPOSE.  See the
% GNU General Public License for more details.

% You should have received a copy of the GNU General Public License
% along with Foobar. If not, see <http://www.gnu.org/licenses/>.
\section{Basis}
Als Basis der Arbeit wurde der quelloffene VPN-Dienst ''strongSwan'' gewählt,
welcher von der Hochschule für Technik Rapperswill entwickelt wird.
Um die Pakete vom Kernelspace in Empfang nehmen zu können, wird der quelloffene
TAP-Gerätetreiber von OpenVPN Technologies, Inc. genutzt. Dieser stellt funktionierende
und nutzbare TUN- und TAP-Geräte bereit, mit denen gearbeitet werden kann.
Für die Verarbeitung der Pakete wird die in ''strongSwan'' integrierte Bibliothek ''libipsec'',
das Plugin ''kernel-libipsec'', sowie bestehende Funktionsheader aus ''libstrongswan''
aus ''strongSwan'' genutzt.

\subsection{Bestehende Implementierungen für Windows}

%native
%Agile Client
%shrewsoft client
%strongswan
Diese Tabelle bezieht sich nur auf die Fähigkeiten, die auf Windows unterstützt sind.
Diese Tabelle macht keine Aussage über die Unterstützung der Software auf anderen Platformen.
\begin{table}[h]
\caption{Vorhandene IPsec-Implementierungen auf Windows }
\begin{tabular*}{20cm}{|p{2cm}|p{2cm}|p{2cm}|p{2cm}|p{2cm}|p{4cm}|p{6cm}|}\firsthline
Name & Version & IKE & ESP & Funktionalität & Sonstiges \\ \hline 
strongswan & 5.5.0 & v1, v2 & \ac{WFP} & MOBIKE, eap-tls, eap-ttls, eap-tnc, RSA-Authentifizierung (unvollständig) & Vollständige Informationen\footnote{\url{https://wiki.strongswan.org/projects/strongswan/wiki/Windows}} \\ \hline 
Windows Agile VPN Client & foo & v1, v2 & Userspace & MOBIKE, ... & hat Probleme beim Installieren der Routen für den ausgehandelten \ac{TS}, kein \ac{DPD}, keine \ac{MFA} möglich, nur schwacher \ac{DH} verfügbar, keine Installation von Routen für IPv6.\footcite{_windows7_2016} \\ \hline 
\end{tabular*}
\label{tab:IPsec-Implementierungen}
\end{table}
\subsection{strongSwan}
''strongSwan'' implementiert einen beträchtlichen Teil der Funktionalität des ''IKEv2''-Standards\footcite{charlie_kaufman_rfc_2014}.
\subsubsection{charon}
\subsubsection{Userspace processing}
\paragraph{OpenVPN TAP-Treiber}
% 255.255.255.255 Netzmaske (Prefix-Länge 32 Bit)
% Route über VPN-Adapter zu entferntem Netzwerk
% Fake ARP
% GW
% Ethernet irr. vom eigentlichen Modus
%async
\paragraph{libipsec}
libipsec ist eine Bibliothek, die IPsec im Tunnelmodus implementiert.
Sie ist Teil des strongSwan-Quellcodes. Sie wird für die Verarbeitung von IP-Paketen
in der Regel auf Systemen genutzt, die keine (funktionierenden) IPsec-Implementierung
enthalten.

libipsec besteht aus zwei Komponenten: Einerseits die Bibliothek, die die SAD und die SPD
implementiert, sowie die Verarbeitung (libipsec) und ein Plugin, welches das Empfangen und das Senden
der Pakete auf dem jeweiligen Betriebssystem implementiert (kernel-libipsec), für das strongSwan kompiliert wurde.

\subparagraph{libipsec}
\subparagraph{kernel-libipsec}
%demultiplexing NON-ESP marker, SPI
\subparagraph{libstrongswan}
libstrongswan ist eine interne Bibliothek von strongSwan, die von den verschiedenen
Versionen von charon geladen wird, um diverse Funktionalität zu erhalten, wie
Linked-Lists, Hashtables, Arrays, sowie die Funktionen um TUN-Geräte zu erstellen und zu öffnen.
% handle_plain
% handle_esp
% callbacks
% synchronous
% WaitForMultipleObjects
% IoCompletionPort inkompatibel mit synchroner I/O -> kompletter Rewrite?
% komplexe Änderungen des Verhaltens von kernel-libipsec

\subsubsection{Kernelspace processing}

