% This file is part of Bachelorarbeit

% Bachelorarbeit is free software: you can redistribute it and/or modify
% it under the terms of the GNU General Public License version 3 as published by
% the Free Software Foundation.

% Bachelorarbeit is distributed in the hope that it will be useful,
% but WITHOUT ANY WARRANTY; without even the implied warranty of
% MERCHANTABILITY or FITNESS FOR A PARTICULAR PURPOSE.  See the
% GNU General Public License for more details.

% You should have received a copy of the GNU General Public License
% along with Foobar. If not, see <http://www.gnu.org/licenses/>.

\chapter{Appendix}
\label{ch:appendix}

Die Daten aus diesen Tabellen wurden unter Nutzung der öffentlich zugänglichen
Dokumente auf den entsprechenden Herstellerwebseiten erstellt.

Wenn bei einem symmetrischen Verschlüsselungsalgorithmus kein Modus angegeben
war, wurde \ac{CBC} angenommen.

Wenn bei einem Authentifizierungsmodus nicht alle unterstützen Permutationen
angegeben waren, wurden alle standardisierten Permutationen als unterstützt 
angenommen.

Wenn ein Feature nicht explizit als unterstützt angegeben wurde, so wurde angenommen
dass es nicht unterstützt wird.

Offenbar ist der ''bintec Secure IPsec Client'' nur ein Rebranding des ''NCP Secure Entry Client'',
wenn man vom \ac{GUI} ausgehen kann. 
Ein weiteres Indiz ist, dass im Installer des ''bintec Secure IPsec Client''
der String ''NCP engineering GmbH'' auftaucht. Daher wäre es zu verstehen,
wenn aus den Dokumentationen der beiden Produkte Featureparität hervorkäme.
Dem ist aber nicht so, wie aus den Tabellen hervorgeht.

Diese Tabellen beziehen sich nur auf die Fähigkeiten, die ab Windows 7 unterstützt sind.
Sie machen keine Aussage über die Unterstützung der Software auf anderen Platformen
und hat keinen Anspruch auf Vollständigkeit.

\paragraph{Symbolik}
\begin{description}
\item[x] Unterstützt
\item[o] Nicht unterstützt
\item[?] unbekannt
\end{description}
\paragraph{Referenzen zur Bibliographie}
\begin{itemize}
\item strongSwan\footcite[][]{_ikev1ciphersuites_2016}\footcite[][]{_ikev2ciphersuites_2016}
\item Windows Agile VPN Client\footcite[][]{_windows7_2016}
\item Shrewsoft VPN Client\footcite[][]{_shrew_2013}
\item NCP Secure Entry Client\footcite[][]{jurgen_honig_datenblatt_2016}
\item bintec Secure IPsec Client\footcite[][]{_bintec_2016-1}
\end{itemize}


\begin{table}[h]
\begin{tabular*}{\textwidth}{|c|c|}\firsthline
Software & IKE-Versionen \\ \hline
strongSwan & IKEv1, IKEv2\\ \hline
Windows Agile VPN Client & IKEv1+L2TP, IKEv2 \\ \hline
Shrewsoft VPN Client & IKEv1 \\ \hline
NCP Secure Entry Client & IKEv1, IKEv2 \\ \hline
bintec Secure IPsec Client & IKEv1, IKEv2 \\ \hline
\end{tabular*}
\label{tab:IPsec-Implementierungen-IKE-Versionen}
\end{table}


\begin{table}[h]
\begin{tabular*}{\textwidth}{|c|c|}\firsthline
Software & Lizenz \\ \hline
strongSwan & MIT/GPLv2 \\ \hline
Windows Agile VPN Client & Proprietär \\ \hline
Shrewsoft VPN Client & Shareware\footnote{\url{https://www.shrew.net/static/help-2.1.x/vpnhelp.htm?WindowsLicense.html}} \\ \hline
NCP Secure Entry Client & Proprietär \\ \hline
bintec Secure IPsec Client & Proprietär \\ \hline
\end{tabular*}
\label{tab:IPsec-Implementierungen-Lizenzen}
\end{table}

\begin{table}[h]
\begin{tabular*}{\textwidth}{|c|c|c|c|c|c|}\firsthline
\backslashbox{Modus}{Software} & strongSwan & Windows & Shrewsoft & NCP & bintec \\ \hline
AES-128-CBC          &  x  & x & x & x & x \\  \hline
AES-192-CBC          &  x  & x & x & x & x \\  \hline
AES-256-CBC          &  x  & x & x & x & x \\  \hline
AES-128-GCM-8        &  x  & o & o & o & o \\  \hline
AES-128-GCM-12       &  x  & o & o & o & o \\  \hline
AES-128-GCM-16       &  x  & o & o & o & o \\  \hline
AES-192-GCM-8        &  x  & o & o & o & o \\  \hline
AES-192-GCM-12       &  x  & o & o & o & o \\  \hline
AES-192-GCM-16       &  x  & o & o & o & o \\  \hline
AES-256-GCM-8        &  x  & o & o & o & o \\  \hline
AES-256-GCM-12       &  x  & o & o & o & o \\  \hline
AES-256-GCM-16       &  x  & o & o & o & o \\  \hline
AES-128-CTR          &  x  & o & o & o & o \\  \hline
AES-192-CTR          &  x  & o & o & o & o \\  \hline
AES-256-CTR          &  x  & o & o & o & o \\  \hline
AES-128-CCM-8        &  x  & o & o & o & o \\  \hline
AES-128-CCM-12       &  x  & o & o & o & o \\  \hline
AES-128-CCM-16       &  x  & o & o & o & o \\  \hline
AES-192-CCM-8        &  x  & o & o & o & o \\  \hline
AES-192-CCM-12       &  x  & o & o & o & o \\  \hline
AES-192-CCM-16       &  x  & o & o & o & o \\  \hline
AES-256-CCM-8        &  x  & o & o & o & o \\  \hline
AES-256-CCM-12       &  x  & o & o & o & o \\  \hline
AES-256-CCM-16       &  x  & o & o & o & o \\  \hline
AES-128-GMAC         &  x  & o & o & o & o \\  \hline
AES-192-GMAC         &  x  & o & o & o & o \\  \hline
AES-256-GMAC         &  x  & o & o & o & o \\  \hline
DES-CBC              &  o  & o & x & o & o \\  \hline
3DES-CBC             &  x  & x & x & x & x \\  \hline
blowfish-128-CBC     &  x  & o & x & x & x \\  \hline
blowfish-192-CBC     &  x  & o & x & x & x \\  \hline
blowfish-256-CBC     &  x  & o & x & x & x \\  \hline
camellia-128-CBC     &  x  & o & o & o & o \\  \hline
camellia-192-CBC     &  x  & o & o & o & o \\  \hline
camellia-256-CBC     &  x  & o & o & o & o \\  \hline
camellia-128-CCM-8   &  x  & o & o & o & o \\  \hline
camellia-128-CCM-12  &  x  & o & o & o & o \\  \hline
camellia-128-CCM-16  &  x  & o & o & o & o \\  \hline
camellia-192-CCM-8   &  x  & o & o & o & o \\  \hline
camellia-192-CCM-12  &  x  & o & o & o & o \\  \hline
camellia-192-CCM-16  &  x  & o & o & o & o \\  \hline
camellia-256-CCM-8   &  x  & o & o & o & o \\  \hline
camellia-256-CCM-12  &  x  & o & o & o & o \\  \hline
camellia-256-CCM-16  &  x  & o & o & o & o \\  \hline
serpent-128-CBC      &  x  & o & o & o & o \\  \hline
serpent-192-CBC      &  x  & o & o & o & o \\  \hline
serpent-256-CBC      &  x  & o & o & o & o \\  \hline
twofish-128-CBC      &  x  & o & o & o & o \\  \hline
twofish-192-CBC      &  x  & o & o & o & o \\  \hline
twofish-256-CBC      &  x  & o & o & o & o \\  \hline
CAST-128-CBC         &  x  & o & x & o & o \\  \hline
chacha20poly1305     &  x  & o & o & o & o \\  \hline
\end{tabular*}
\label{tab:IPsec-Implementierungen-Vertraulichkeit-Algorithmen}
\end{table}

\begin{table}[h]
\begin{tabular*}{\textwidth}{|c|c|c|c|c|c|}\firsthline
\backslashbox{Modus}{Software} & strongSwan & Windows & Shrewsoft & NCP & bintec \\ \hline
MD5                                                     & x & o & x & x & x \\  \hline
SHA-1                                                   & x & x & x & o & x \\  \hline
SHA-256                                                 & x & x & o & x & x \\  \hline
SHA-384                                                 & x & x & o & x & x \\  \hline
SHA-512                                                 & x & o & o & x & x \\  \hline
SHA-256-96\footnote{SHA-256 mit Truncation auf 96 Bit}  & x & x & o & o & o \\  \hline
AES-XCBC                                                & x & o & o & o & o \\  \hline
AES-128-GMAC                                            & x & o & o & o & o \\  \hline
AES-192-GMAC                                            & x & o & o & o & o \\  \hline
AES-256-GMAC                                            & x & o & o & o & o \\  \hline

\end{tabular*}
\label{tab:IPsec-Implementierungen-Authentizitaet-Algorithmen}
\end{table}

\begin{table}[h]
\begin{tabular*}{\textwidth}{|c|c|c|c|c|c|}\firsthline
\backslashbox{Modus}{Software} & strongSwan & Windows & Shrewsoft & NCP & bintec \\ \hline
MODP-768       & x & o & x & x & x \\  \hline
MODP-1024      & x & x & x & x & x \\  \hline
MODP-1536      & x & o & x & x & x \\  \hline
MODP-2048      & x & o & x & x & x \\  \hline
MODP-3072      & x & o & x & x & x \\  \hline
MODP-4096      & x & o & x & x & x \\  \hline
MODP-6144      & x & o & x & x & x \\  \hline
MODP-8192      & x & o & x & x & x \\  \hline
MODP-1024s160  & x & o & o & o & o \\  \hline
MODP-2048s224  & x & o & o & o & o \\  \hline
MODP-2048s256  & x & o & o & o & o \\  \hline
ECP-192        & x & o & o & x & o \\  \hline
ECP-224        & x & o & o & x & o \\  \hline
ECP-256        & x & o & o & x & o \\  \hline
ECP-384        & x & o & o & x & o \\  \hline
ECP-521        & x & o & o & x & o \\  \hline
ECP-224BP      & x & o & o & o & o \\  \hline
ECP-256BP      & x & o & o & o & o \\  \hline
ECP-384BP      & x & o & o & o & o \\  \hline
ECP-512BP      & x & o & o & o & o \\  \hline
NTRU-112       & x & o & o & o & o \\  \hline
NTRU-128       & x & o & o & o & o \\  \hline
NTRU-192       & x & o & o & o & o \\  \hline
NTRU-256       & x & o & o & o & o \\  \hline
NEWHOPE-128    & x & o & o & o & o \\  \hline
\end{tabular*}
\label{tab:IPsec-Implementierungen-DH-Algorithmen}
\end{table}

\begin{table}[h]
\begin{tabular*}{\textwidth}{|c|c|c|c|c|c|}\firsthline
\backslashbox{Modus}{Software} & strongSwan & Windows & Shrewsoft & NCP & bintec                  \\ \hline
Hybrid\footnote{Client mit XAUTH, Server mittels X.509}  & x & x & x & x & x  \\ \hline
PSK                                                      & x & o & x & x & o  \\ \hline
PSK + XAUTH                                              & x & o & x & x & o  \\ \hline
X.509                                                    & x & x & x & x & x  \\ \hline
EAP-MD5                                                  & x & o & o & x & o  \\ \hline
EAP-PAP                                                  & x & o & o & x & o  \\ \hline
EAP-CHAP                                                 & x & o & o & x & o  \\ \hline
EAP-MSCHAPv2                                             & x & x & o & x & o  \\ \hline
EAP-GTC                                                  & x & o & o & o & o  \\ \hline
EAP-TLS                                                  & x & x & o & x & o  \\ \hline
EAP-TTLS                                                 & x & o & o & o & o  \\ \hline
EAP-AKA                                                  & x & o & o & o & o  \\ \hline
EAP-TNC                                                  & x & o & o & o & o  \\ \hline
TNC-IMC                                                  & x & o & o & o & o  \\ \hline
TNC-IMV                                                  & x & o & o & o & o  \\ \hline

\end{tabular*}
\label{tab:IPsec-Implementierungen-Authentifizierungs-Modi}
\end{table}

\begin{table}[h]
\begin{tabular*}{\textwidth}{|c|c|c|c|c|c|}\firsthline
\backslashbox{Modus}{Software} & strongSwan & Windows & Shrewsoft & NCP & bintec \\ \hline
CRL  & x & x & o & x & x \\ \hline
OCSP & x & ? & o & x & x \\ \hline
\end{tabular*}
\label{tab:IPsec-Implementierungen-CRL-Support}
\end{table}


\begin{table}[h]
\begin{tabular*}{\textwidth}{|c|c|c|c|c|c|}\firsthline
\backslashbox{Modus}{Software} & strongSwan & Windows & Shrewsoft & NCP & bintec \\ \hline
Tunnel-Modus     & x & x & x & x & x \\  \hline
Transport-Modus  & x & x & o & o & o \\  \hline
BEET-Modus       & x & o & o & o & o \\  \hline
Komprimierung    & x & o & o & x & o \\ \hline
\end{tabular*}
\label{tab:IPsec-Implementierungen-Tunnel-Modi}
\end{table}

\begin{table}[h]
\begin{tabular*}{\textwidth}{|c|c|c|c|c|c|}\firsthline
\backslashbox{Modus}{Software} & strongSwan & Windows & Shrewsoft & NCP & bintec \\ \hline
Main Mode       & x & x & x & x & ? \\ \hline
Aggressive Mode & x & x & x & x & ? \\ \hline 
Quick Mode      & x & o & x & x & ? \\ \hline
Config Mode     & x & x & x & x & x \\ \hline
\end{tabular*}
\label{tab:IPsec-Implementierungen-IKE-Modi}
\end{table}

\begin{table}[h]
\begin{tabular*}{\textwidth}{|c|c|c|c|c|c|}\firsthline
\backslashbox{Feature}{Software} & strongSwan & Windows & Shrewsoft & NCP & bintec \\ \hline
NAT-T                 & x & x                     & x & x & x \\ \hline
DPD                   & x & x                     & x & x & x \\ \hline
MOBIKE                & x & x                     & o & o & o \\ \hline
GUI                   & o & x                     & x & x & x \\ \hline
IPsec über TCP        & o & o                     & o & x & x \\ \hline
IPv6                  & x & x                     & x & x & x \\ \hline
Attribut-Zertifikate  & x & ?                     & o & o & ? \\ \hline
PFS                   & x & o                     & x & x & x \\ \hline
IKE-Fragmentierung    & x & ?\footnote{Nur IKEv1} & x & o & o \\ \hline

\end{tabular*}
\label{tab:IPsec-Implementierungen-Features}
\end{table}
