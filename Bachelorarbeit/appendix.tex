% This file is part of Bachelorarbeit

% Bachelorarbeit is free software: you can redistribute it and/or modify
% it under the terms of the GNU General Public License version 3 as published by
% the Free Software Foundation.

% Bachelorarbeit is distributed in the hope that it will be useful,
% but WITHOUT ANY WARRANTY; without even the implied warranty of
% MERCHANTABILITY or FITNESS FOR A PARTICULAR PURPOSE.  See the
% GNU General Public License for more details.

% You should have received a copy of the GNU General Public License
% along with Foobar. If not, see <http://www.gnu.org/licenses/>.

\section{Appendix}
\label{sec:appendix}

\subsection{Featurematrix}
\label{subsec:featurematrix}
Die Daten aus diesen Tabellen wurden unter Nutzung der öffentlich zugänglichen
Dokumente auf den entsprechenden Herstellerwebseiten erstellt.

Wenn bei einem symmetrischen Verschlüsselungsalgorithmus kein Modus angegeben
war, wurde \ac{CBC} angenommen.

Wenn bei einem Authentifizierungsmodus nicht alle unterstützen Permutationen
angegeben waren, wurden alle standardisierten Permutationen als unterstützt 
angenommen.

Wenn ein Feature nicht explizit als unterstützt angegeben wurde, so wurde angenommen
dass es nicht unterstützt wird.

Offenbar ist der ''bintec Secure IPsec Client'' nur ein Rebranding des ''NCP Secure Entry Client'',
wenn man vom \ac{GUI} ausgehen kann. 
Ein weiteres Indiz ist, dass im Installer des ''bintec Secure IPsec Client''
der String ''NCP engineering GmbH'' auftaucht. Daher wäre es zu verstehen,
wenn aus den Dokumentationen der beiden Produkte Featureparität hervorkäme.
Dem ist aber nicht so, wie aus den Tabellen hervorgeht.

Diese Tabellen beziehen sich nur auf die Fähigkeiten, die ab Windows 7 unterstützt sind.
Sie machen keine Aussage über die Unterstützung der Software auf anderen Platformen
und hat keinen Anspruch auf Vollständigkeit.

\paragraph{Symbolik}
\begin{description}
\item[x] Unterstützt
\item[o] Nicht unterstützt
\item[?] unbekannt
\end{description}
\paragraph{Referenzen zur Bibliographie}
\begin{itemize}
\item strongSwan\footcite[][]{_ikev1ciphersuites_2016}\footcite[][]{_ikev2ciphersuites_2016}
\item Windows Agile VPN Client\footcite[][]{_windows7_2016}
\item Shrewsoft VPN Client\footcite[][]{_shrew_2013}
\item NCP Secure Entry Client\footcite[][]{jurgen_honig_datenblatt_2016}
\item bintec Secure IPsec Client\footcite[][]{_bintec_2016-1}
\end{itemize}

\begin{center}
\begin{table}[h!]
\begin{tabularx}{\textwidth}{|X|X|}\firsthline
Software & IKE-Versionen \\ \hline
strongSwan & IKEv1, IKEv2\\ \hline
Windows Agile VPN Client & IKEv1+L2TP, IKEv2 \\ \hline
Shrewsoft VPN Client & IKEv1 \\ \hline
NCP Secure Entry Client & IKEv1, IKEv2 \\ \hline
bintec Secure IPsec Client & IKEv1, IKEv2 \\ \hline
\end{tabularx}
\label{tab:IPsec-Implementierungen-IKE-Versionen}
\caption{Unterstützte IKE-Versionen der IPsec-Implementierungen}
\end{table}


\begin{table}[h!]
\begin{tabularx}{\textwidth}{|X|X|}\firsthline
Software & Lizenz \\ \hline
strongSwan & MIT/GPLv2 \\ \hline
Windows Agile VPN Client & Proprietär \\ \hline
Shrewsoft VPN Client & Shareware \\ \hline
NCP Secure Entry Client & Proprietär \\ \hline
bintec Secure IPsec Client & Proprietär \\ \hline
\end{tabularx}
\label{tab:IPsec-Implementierungen-Lizenzen}
\caption{Lizenzen der IPsec-Implementierungen}
\end{table}

\begin{table}[h!]
\begin{tabularx}{\textwidth}{|X|c|c|c|c|c|}\firsthline
\backslashbox{Modus}{Software} & strongSwan & Windows & Shrewsoft & NCP & bintec \\ \hline
AES-128-CBC          &  x  & x & x & x & x \\  \hline
AES-192-CBC          &  x  & x & x & x & x \\  \hline
AES-256-CBC          &  x  & x & x & x & x \\  \hline
AES-128-GCM-8        &  x  & o & o & o & o \\  \hline
AES-128-GCM-12       &  x  & o & o & o & o \\  \hline
AES-128-GCM-16       &  x  & o & o & o & o \\  \hline
AES-192-GCM-8        &  x  & o & o & o & o \\  \hline
AES-192-GCM-12       &  x  & o & o & o & o \\  \hline
AES-192-GCM-16       &  x  & o & o & o & o \\  \hline
AES-256-GCM-8        &  x  & o & o & o & o \\  \hline
AES-256-GCM-12       &  x  & o & o & o & o \\  \hline
AES-256-GCM-16       &  x  & o & o & o & o \\  \hline
AES-128-CTR          &  x  & o & o & o & o \\  \hline
AES-192-CTR          &  x  & o & o & o & o \\  \hline
AES-256-CTR          &  x  & o & o & o & o \\  \hline
AES-128-CCM-8        &  x  & o & o & o & o \\  \hline
AES-128-CCM-12       &  x  & o & o & o & o \\  \hline
AES-128-CCM-16       &  x  & o & o & o & o \\  \hline
AES-192-CCM-8        &  x  & o & o & o & o \\  \hline
AES-192-CCM-12       &  x  & o & o & o & o \\  \hline
AES-192-CCM-16       &  x  & o & o & o & o \\  \hline
AES-256-CCM-8        &  x  & o & o & o & o \\  \hline
AES-256-CCM-12       &  x  & o & o & o & o \\  \hline
AES-256-CCM-16       &  x  & o & o & o & o \\  \hline
DES-CBC              &  o  & o & x & o & o \\  \hline
3DES-CBC             &  x  & x & x & x & x \\  \hline
BLOWFISH-128-CBC     &  x  & o & x & x & x \\  \hline
BLOWFISH-192-CBC     &  x  & o & x & x & x \\  \hline
BLOWFISH-256-CBC     &  x  & o & x & x & x \\  \hline
CAMELLIA-128-CBC     &  x  & o & o & o & o \\  \hline
CAMELLIA-192-CBC     &  x  & o & o & o & o \\  \hline
CAMELLIA-256-CBC     &  x  & o & o & o & o \\  \hline
CAMELLIA-128-CCM-8   &  x  & o & o & o & o \\  \hline
CAMELLIA-128-CCM-12  &  x  & o & o & o & o \\  \hline
CAMELLIA-128-CCM-16  &  x  & o & o & o & o \\  \hline
CAMELLIA-192-CCM-8   &  x  & o & o & o & o \\  \hline
CAMELLIA-192-CCM-12  &  x  & o & o & o & o \\  \hline
CAMELLIA-192-CCM-16  &  x  & o & o & o & o \\  \hline
CAMELLIA-256-CCM-8   &  x  & o & o & o & o \\  \hline
CAMELLIA-256-CCM-12  &  x  & o & o & o & o \\  \hline
CAMELLIA-256-CCM-16  &  x  & o & o & o & o \\  \hline
SERPENT-128-CBC      &  x  & o & o & o & o \\  \hline
SERPENT-192-CBC      &  x  & o & o & o & o \\  \hline
SERPENT-256-CBC      &  x  & o & o & o & o \\  \hline
TWOFISH-128-CBC      &  x  & o & o & o & o \\  \hline
TWOFISH-192-CBC      &  x  & o & o & o & o \\  \hline
TWOFISH-256-CBC      &  x  & o & o & o & o \\  \hline
CAST-128-CBC         &  x  & o & x & o & o \\  \hline
chacha20poly1305     &  x  & o & o & o & o \\  \hline
\end{tabularx}
\label{tab:IPsec-Implementierungen-Vertraulichkeit-Algorithmen}
\caption{Unterstützte Algorithmen für Vertraulichkeit der IPsec-Implementierungen}
\end{table}

\begin{table}[h!]
\begin{tabularx}{\textwidth}{|X|c|c|c|c|c|}\firsthline
\backslashbox{Modus}{Software} & strongSwan & Windows & Shrewsoft & NCP & bintec \\ \hline
MD5                                                     & x & o & x & x & x \\  \hline
SHA-1                                                   & x & x & x & o & x \\  \hline
SHA-256                                                 & x & x & o & x & x \\  \hline
SHA-384                                                 & x & x & o & x & x \\  \hline
SHA-512                                                 & x & o & o & x & x \\  \hline
SHA-256-96                                              & x & x & o & o & o \\  \hline
AES-XCBC                                                & x & o & o & o & o \\  \hline
AES-128-GMAC                                            & x & o & o & o & o \\  \hline
AES-192-GMAC                                            & x & o & o & o & o \\  \hline
AES-256-GMAC                                            & x & o & o & o & o \\  \hline
\end{tabularx}
\label{tab:IPsec-Implementierungen-Authentizitaet-Algorithmen}
\caption{Unterstützte Algorithmen für Authentizität der IPsec-Implementierungen}
\end{table}

\begin{table}[h!]
\begin{tabularx}{\textwidth}{|X|c|c|c|c|c|}\firsthline
\backslashbox{Modus}{Software} & strongSwan & Windows & Shrewsoft & NCP & bintec \\ \hline
MODP-768       & x & o & x & x & x \\  \hline
MODP-1024      & x & x & x & x & x \\  \hline
MODP-1536      & x & o & x & x & x \\  \hline
MODP-2048      & x & x & x & x & x \\  \hline
MODP-3072      & x & o & x & x & x \\  \hline
MODP-4096      & x & o & x & x & x \\  \hline
MODP-6144      & x & o & x & x & x \\  \hline
MODP-8192      & x & o & x & x & x \\  \hline
MODP-1024s160  & x & o & o & o & o \\  \hline
MODP-2048s224  & x & o & o & o & o \\  \hline
MODP-2048s256  & x & o & o & o & o \\  \hline
ECP-192        & x & o & o & x & o \\  \hline
ECP-224        & x & o & o & x & o \\  \hline
ECP-256        & x & o & o & x & o \\  \hline
ECP-384        & x & o & o & x & o \\  \hline
ECP-521        & x & o & o & x & o \\  \hline
ECP-224BP      & x & o & o & o & o \\  \hline
ECP-256BP      & x & o & o & o & o \\  \hline
ECP-384BP      & x & o & o & o & o \\  \hline
ECP-512BP      & x & o & o & o & o \\  \hline
NTRU-112       & x & o & o & o & o \\  \hline
NTRU-128       & x & o & o & o & o \\  \hline
NTRU-192       & x & o & o & o & o \\  \hline
NTRU-256       & x & o & o & o & o \\  \hline
NEWHOPE-128    & x & o & o & o & o \\  \hline
\end{tabularx}
\label{tab:IPsec-Implementierungen-DH-Algorithmen}
\caption{Unterstützte Schlüsselaustauschprotokolle der IPsec-Implementierungen}
\end{table}

\begin{table}[h!]
\begin{tabularx}{\textwidth}{|X|c|c|c|c|c|}\firsthline
\backslashbox{Modus}{Software} & strongSwan & Windows & Shrewsoft & NCP & bintec                  \\ \hline
Hybrid                                                   & x & x & x & x & x  \\ \hline
PSK                                                      & x & o & x & x & o  \\ \hline
PSK + XAUTH                                              & x & o & x & x & o  \\ \hline
X.509                                                    & x & x & x & x & x  \\ \hline
EAP-MD5                                                  & x & o & o & x & o  \\ \hline
EAP-PAP                                                  & x & o & o & x & o  \\ \hline
EAP-CHAP                                                 & x & o & o & x & o  \\ \hline
EAP-MSCHAPv2                                             & x & x & o & x & o  \\ \hline
EAP-GTC                                                  & x & o & o & o & o  \\ \hline
EAP-TLS                                                  & x & x & o & x & o  \\ \hline
EAP-TTLS                                                 & x & o & o & o & o  \\ \hline
EAP-AKA                                                  & x & o & o & o & o  \\ \hline
EAP-TNC                                                  & x & o & o & o & o  \\ \hline
TNC-IMC                                                  & x & o & o & o & o  \\ \hline
TNC-IMV                                                  & x & o & o & o & o  \\ \hline
\end{tabularx}
\label{tab:IPsec-Implementierungen-Authentifizierungs-Modi}
\caption{Unterstützte Authentifizierungsmethoden der IPsec-Implementierungen}
\end{table}

\begin{table}[h!]
\begin{tabularx}{\textwidth}{|X|c|c|c|c|c|}\firsthline
\backslashbox{Modus}{Software} & strongSwan & Windows & Shrewsoft & NCP & bintec \\ \hline
CRL  & x & x & o & x & x \\ \hline
OCSP & x & ? & o & x & x \\ \hline
\end{tabularx}
\label{tab:IPsec-Implementierungen-CRL-Support}
\caption{Unterstützte Mechanismen zum Zurückziehen von Zertifikaten der IPsec-Implementierungen}
\end{table}


\begin{table}[h!]
\begin{tabularx}{\textwidth}{|X|c|c|c|c|c|}\firsthline
\backslashbox{Modus}{Software} & strongSwan & Windows & Shrewsoft & NCP & bintec \\ \hline
Tunnel-Modus     & x & x & x & x & x \\  \hline
Transport-Modus  & x & x & o & o & o \\  \hline
BEET-Modus       & x & o & o & o & o \\  \hline
\end{tabularx}
\label{tab:IPsec-Implementierungen-Tunnel-Modi}
\caption{Unterstützte Tunnel-Modi der IPsec-Implementierungen}
\end{table}

\begin{table}[h!]
\begin{tabularx}{\textwidth}{|X|c|c|c|c|c|}\firsthline
\backslashbox{Modus}{Software} & strongSwan & Windows & Shrewsoft & NCP & bintec \\ \hline
Main Mode       & x & x & x & x & ? \\ \hline
Aggressive Mode & x & x & x & x & ? \\ \hline 
Quick Mode      & x & o & x & x & ? \\ \hline
Config Mode     & x & x & x & x & x \\ \hline
\end{tabularx}
\label{tab:IPsec-Implementierungen-IKE-Modi}
\caption{Unterstützte IKE-Modi der IPsec-Implementierungen}
\end{table}

\begin{table}[h!]
\begin{tabularx}{\textwidth}{|X|c|c|c|c|c|}\firsthline
\backslashbox{Feature}{Software} & strongSwan & Windows & Shrewsoft & NCP & bintec \\ \hline
NAT-T                 & x & x                     & x & x & x \\ \hline
DPD                   & x & x                     & x & x & x \\ \hline
MOBIKE                & x & x                     & o & o & o \\ \hline
GUI                   & o & x                     & x & x & x \\ \hline
IPsec über TCP        & o & o                     & o & x & x \\ \hline
IPv6                  & x & x                     & x & x & x \\ \hline
Attribut-Zertifikate  & x & ?                     & o & o & ? \\ \hline
PFS                   & x & o                     & x & x & x \\ \hline
IKE-Fragmentierung    & x & Nur IKEv1             & x & o & o \\ \hline
Komprimierung         & x & o                     & o & x & o \\ \hline
\end{tabularx}
\label{tab:IPsec-Implementierungen-Features}
\caption{Unterstützte Features der IPsec-Implementierungen}
\end{table}
\end{center}

\subsection{Testkonfiguration}
\label{subsec:Testkonfiguration}

\begin{lstlisting}[caption=Testkonfiguration - swanctl.conf,label=lst:swanctl.conf]
connections {
        foo {
            version = 2
            dpd_delay = 10
            dpd_timeout = 60
            fragmentation = yes
            send_cert = always
            remote_addrs = 37.120.161.220
            proposals = aes256gcm16-prfsha256-modp4096
            vips = 0.0.0.0
			mobike = no
			encap = yes
            local {
                auth = eap-tls
                certs = certificate.pem
            }
            remote {
                auth = pubkey
                id = thermi.strangled.net
                cacerts = serverca.pem
            }
            children {
                    bar {
                        dpd_action = restart
                        start_action = none
                        esp_proposals = aes256-sha256-ecp521
                        remote_ts = 172.16.25.2/32
                    }
            }
        }
}
\end{lstlisting}

\begin{lstlisting}[caption=Testkonfiguration - strongswan.conf,label=lst:strongswan.conf]
charon-svc {
	filelog {
		this_log.txt{
			default=2
			job = 1
			mgr = 0
			enc = 0 
			asn = 0
			flush_line = yes
			ike_name = yes
			append = no
		}
	}
}
\end{lstlisting}

\begin{lstlisting}[caption=Code von win32.h,label=lst:libstrongswan-win32.h]
/*
 * Copyright (C) 2016 Noel Kuntze
 *
 * Permission is hereby granted, free of charge, to any person obtaining a copy
 * of this software and associated documentation files (the "Software"), to deal
 * in the Software without restriction, including without limitation the rights
 * to use, copy, modify, merge, publish, distribute, sublicense, and/or sell
 * copies of the Software, and to permit persons to whom the Software is
 * furnished to do so, subject to the following conditions:
 *
 * The above copyright notice and this permission notice shall be included in
 * all copies or substantial portions of the Software.
 *
 * THE SOFTWARE IS PROVIDED "AS IS", WITHOUT WARRANTY OF ANY KIND, EXPRESS OR
 * IMPLIED, INCLUDING BUT NOT LIMITED TO THE WARRANTIES OF MERCHANTABILITY,
 * FITNESS FOR A PARTICULAR PURPOSE AND NONINFRINGEMENT. IN NO EVENT SHALL THE
 * AUTHORS OR COPYRIGHT HOLDERS BE LIABLE FOR ANY CLAIM, DAMAGES OR OTHER
 * LIABILITY, WHETHER IN AN ACTION OF CONTRACT, TORT OR OTHERWISE, ARISING FROM,
 * OUT OF OR IN CONNECTION WITH THE SOFTWARE OR THE USE OR OTHER DEALINGS IN
 * THE SOFTWARE.
 */

#ifndef WIN32_H
#define WIN32_H

#define WIN32_TUN_READ_EVENT_TEMPLATE "WIN32-libipsec-read-device-%d"
#define WIN32_TUN_WRITE_EVENT_TEMPLATE "WIN32-libipsec-write-device-%d"
#define WIN32_TUN_EVENT_LENGTH 80
#define TAP_WIN_COMPONENT_ID "tap0901"

#define ADAPTER_KEY "SYSTEM\\CurrentControlSet\\Control\\Class\\{4D36E972-E325-11CE-BFC1-08002BE10318}"
#define NETWORK_CONNECTIONS_KEY "SYSTEM\\CurrentControlSet\\Control\\Network\\{4D36E972-E325-11CE-BFC1-08002BE10318}"

/*
 * ======================
 * Filesystem prefixes
 * ======================
 */

#define USERMODEDEVICEDIR "\\\\.\\Global\\"
#define SYSDEVICEDIR      "\\Device\\"
#define USERDEVICEDIR     "\\DosDevices\\Global\\"
#define TAP_WIN_SUFFIX    ".tap"

/*
 * TAP IOCTL constants and macros.
 *
 */
#define TAP_WIN_CONTROL_CODE(request,method) \
  CTL_CODE (FILE_DEVICE_UNKNOWN, request, method, FILE_ANY_ACCESS)

/* Present in 8.1 */

#define TAP_WIN_IOCTL_GET_MAC               TAP_WIN_CONTROL_CODE (1, METHOD_BUFFERED)
#define TAP_WIN_IOCTL_GET_VERSION           TAP_WIN_CONTROL_CODE (2, METHOD_BUFFERED)
#define TAP_WIN_IOCTL_GET_MTU               TAP_WIN_CONTROL_CODE (3, METHOD_BUFFERED)
#define TAP_WIN_IOCTL_GET_INFO              TAP_WIN_CONTROL_CODE (4, METHOD_BUFFERED)
#define TAP_WIN_IOCTL_CONFIG_POINT_TO_POINT TAP_WIN_CONTROL_CODE (5, METHOD_BUFFERED)
#define TAP_WIN_IOCTL_SET_MEDIA_STATUS      TAP_WIN_CONTROL_CODE (6, METHOD_BUFFERED)
#define TAP_WIN_IOCTL_CONFIG_DHCP_MASQ      TAP_WIN_CONTROL_CODE (7, METHOD_BUFFERED)
#define TAP_WIN_IOCTL_GET_LOG_LINE          TAP_WIN_CONTROL_CODE (8, METHOD_BUFFERED)
#define TAP_WIN_IOCTL_CONFIG_DHCP_SET_OPT   TAP_WIN_CONTROL_CODE (9, METHOD_BUFFERED)

/* Added in 8.2 */

/* obsoletes TAP_WIN_IOCTL_CONFIG_POINT_TO_POINT */
#define TAP_WIN_IOCTL_CONFIG_TUN            TAP_WIN_CONTROL_CODE (10, METHOD_BUFFERED)
#define TAP_WIN_IOCTL_CONFIG_SET_SRC_CHECK  TAP_WIN_CONTROL_CODE (11, METHOD_BUFFERED)


#endif /* WIN32_H */
\end{lstlisting}

\begin{lstlisting}[caption=Code für das Suchen eines TAP-Geräts,label=lst:find_tap_devices]
/*
 * Searches through the registry for suitable TAP driver interfaces
 * On Windows, the TAP interface metadata is stored and described in the registry.
 * It returns a linked list that contains all found guids. The guids describe the interfaces.
 */

linked_list_t *find_tap_devices()
{
    char enum_name[256], unit_string[256],
    instance_id[256], component_id[256],
    component_id_string[] = "ComponentId",
    instance_id_string[] = "NetCfgInstanceId";
    LONG status;
    uint32_t i = 0;
    DWORD len, type;
    HKEY adapter_key, unit_key;
    linked_list_t *list = linked_list_create();

    /*
     * Open parent key. It contains all other keys that
     * describe any possible interfaces.
     */
    status = RegOpenKeyEx(
            HKEY_LOCAL_MACHINE,
            ADAPTER_KEY,
            0,
            KEY_READ,
            &adapter_key);

    if (status == ERROR_SUCCESS)
    {
        while (TRUE)
        {
            len = sizeof (enum_name);
            status = RegEnumKeyEx(
                    adapter_key,
                    i,
                    enum_name,
                    &len,
                    NULL,
                    NULL,
                    NULL,
                    NULL);
            if (status == ERROR_SUCCESS)
            {
                snprintf(unit_string, sizeof (unit_string), "%s\\%s",
                        ADAPTER_KEY, enum_name);

                status = RegOpenKeyEx(
                        HKEY_LOCAL_MACHINE,
                        unit_string,
                        0,
                        KEY_READ,
                        &unit_key);

                if (status == ERROR_SUCCESS)
                {
                    len = sizeof (component_id);
                    status = RegQueryValueEx(
                            unit_key,
                            component_id_string,
                            NULL,
                            &type,
                            component_id,
                            &len);

                    if (status == ERROR_SUCCESS && type == REG_SZ)
                    {
                        len = sizeof (instance_id);
                        status = RegQueryValueEx(
                                unit_key,
                                instance_id_string,
                                NULL,
                                &type,
                                instance_id,
                                &len);

                        if (status == ERROR_SUCCESS && type == REG_SZ)
                        {
                            if (!strcmp(component_id, TAP_WIN_COMPONENT_ID))
                            {
                                /* That thing is a valid interface key */
                                /* link into return list */
                                char *guid = malloc(sizeof(instance_id));
                                memcpy(guid, instance_id, sizeof(instance_id));
                                list->insert_last(list, guid);
                            }
                        }
                    }
                    else
                    {
                        DBG2(DBG_LIB, "Error opening registry key: %s\\%s",
                                unit_string, component_id_string);
                    }
                    RegCloseKey(unit_key);
                }
                else if (status != ERROR_SUCCESS)
                {
                    DBG2(DBG_LIB, "Error opening registry key: %s", unit_string);
                }
                i++;
            }
            else if (status == ERROR_NO_MORE_ITEMS)
            {
                break;
            }
            else
            {
                DBG2(DBG_LIB, "Error enumerating registry subkeys of key: %s",
                        ADAPTER_KEY);
            }
        }
    }
    else
    {
        DBG2(DBG_LIB, "Error opening registry key: %s", ADAPTER_KEY);
    }

    RegCloseKey(adapter_key);
    return list;
}
\end{lstlisting}


\begin{lstlisting}[caption=Code für handle\_plain auf Windows,label=lst:handle-plain-windows]
/**
 * Job handling outbound plaintext packets
 */
static job_requeue_t handle_plain(private_kernel_libipsec_router_t *this)
{
#ifdef WIN32
        void **key = NULL;
        bool oldstate;
        uint32_t length, event_status = 0, i = 0, j = 0, offset;
        handle_overlapped_buffer_t *bundle_array = NULL, dummy, tun_device_handle_overlapped_buffer;
        OVERLAPPED *overlapped = NULL;
        HANDLE *event_array = NULL, tun_device_event;
        tun_device_t *tun_device = this->tun.tun;
        enumerator_t *tuns_enumerator;

        memset(&tun_device_handle_overlapped_buffer, 0, sizeof(handle_overlapped_buffer_t));
        /* Reset synchronisation event */
        ResetEvent(this->event);

        length = this->tuns->get_count(this->tuns);

        this->lock->read_lock(this->lock);
        /* Read event for this->tun */

        /* allocate arrays for all the structs we need */
        /* events, overlapped structures and bundles. */
        /* event_array holds all the HANDLE structures for the events that are
         * used for notifying the thread of finished reads and writes.
         */

        overlapped = alloca((length+2)*sizeof(OVERLAPPED));
        event_array = alloca((length+2)*sizeof(HANDLE));
        bundle_array = alloca((length+2)*sizeof(handle_overlapped_buffer_t));

        memset(overlapped, 0, (length+2)*sizeof(OVERLAPPED));
        memset(bundle_array, 0, (length+2)*sizeof(handle_overlapped_buffer_t));

        /* These are the arrays we're going to work with */

        /* first position is the event we use for synchronisation  */
        /* Insert notification event */
        event_array[i] = this->event;
        /* Insert dummy structure */
        bundle_array[i] = dummy;
        i++;

        /* second position is this->tun */
        /* insert event object for this->tun device */
        tun_device_event = CreateEvent(NULL, FALSE, FALSE, FALSE);
        if (!tun_device_event)
        {
            char *error_message = format_error(GetLastError());
            free(error_message);
            return JOB_REQUEUE_FAIR;
        }
        event_array[i] = tun_device_event;
        ResetEvent(event_array[i]);
        /* bundle for the read on this->tun */
        /* Reserve memory for the buffer*/
        tun_device_handle_overlapped_buffer.buffer = chunk_alloca(tun_device->get_mtu(tun_device));
        /* Initialise the buffer */
        memset(tun_device_handle_overlapped_buffer.buffer.ptr, 0, tun_device_handle_overlapped_buffer.buffer.len);

        tun_device_handle_overlapped_buffer.fileHandle = tun_device->get_handle(tun_device);
        tun_device_handle_overlapped_buffer.overlapped = overlapped;

        tun_device_handle_overlapped_buffer.overlapped->hEvent= tun_device_event;

        bundle_array[i] = tun_device_handle_overlapped_buffer;

        i++;

        /* Start ReadFile for this->tun.handle */
        if (!start_read(&tun_device_handle_overlapped_buffer, tun_device_handle_overlapped_buffer.overlapped->hEvent))
        {
                // TODO: Cleanup heap
                this->lock->unlock(this->lock);
                return JOB_REQUEUE_FAIR;
        }
        /* pad bundle_array with two empty structures */
        /* iterate over all our tun devices, create event handles, reset them, queue read operations on all handles */


        tuns_enumerator = this->tuns->create_enumerator(this->tuns);
        while(tuns_enumerator->enumerate(tuns_enumerator, key, &tun_device))
        {
            /* Allocate structure and buffer */

            bundle_array[i].buffer = chunk_alloca(tun_device->get_mtu(tun_device));
            memset(bundle_array[i].buffer.ptr, 0, bundle_array[i].buffer.len);
            bundle_array[i].fileHandle = tun_device->get_handle(tun_device);
            /* Allocate and initialise OVERLAPPED structure */
            bundle_array[i].overlapped = alloca(sizeof(OVERLAPPED));
            (*bundle_array[i].overlapped) = overlapped[i];
            memset(&bundle_array[i].overlapped, 0, sizeof(OVERLAPPED));
            /* Create unique name for that event. */
            /* Create unique event for read accesses on that device
             * No security attributes, no manual reset, initial state is unsignaled,
             * name is the special name we created
             */
            bundle_array[i].overlapped->hEvent = CreateEvent(NULL, FALSE, FALSE, FALSE);
            // event_array[i] = OpenEvent(EVENT_ALL_ACCESS, FALSE, tun_device->get_read_event_name(tun_device));
            event_array[i] = bundle_array[i].overlapped->hEvent;

            if (event_array[i] == NULL)
            {
                char *error_message = format_error(GetLastError());
                free(error_message);
                return JOB_REQUEUE_FAIR;
            }
            i++;

            /* Initialise read with the allocate overwrite structure */
            DBG2(DBG_ESP, "Reading on %s", tun_device->get_name(tun_device));
            if (!start_read(&bundle_array[i], bundle_array[i].overlapped->hEvent))
            {
                    // TODO: Cleanup heap
                    this->lock->unlock(this->lock);
                    return JOB_REQUEUE_FAIR;
            }
            i++;
        }
        tuns_enumerator->destroy(tuns_enumerator);

        while (TRUE)
        {
            /* Wait for a handle to be signaled */
            /* In the mingw64 sources, MAXIMUM_WAIT_OBJECTS is defined as 64. That means we can wait for a maximum of 64 event handles.
             * This translates to 63 tun devices. I think this is sufficiently high to not have to implement a mechanism for waiting for more
             * events /support more TUN devices */
            oldstate = thread_cancelability(FALSE);
            event_status = WaitForMultipleObjects(i, event_array, FALSE, INFINITE);
            thread_cancelability(oldstate);
            offset = event_status - WAIT_OBJECT_0;

            /* A handle was signaled. Find the tun handle whose read was successful */

            /* We can only use the event_status of indication for the first completed IO operation.
             * After the event was signaled, we need to test the OVERLAPPED structure in the other array
             * to find out what event was signaled.
             */
            /*
             * Probably broken?
             */
            /* Check if an event in the array was signaled. (That is the case if
             * the event_status is between WAIT_OBJECT_0 and WAIT_OBJECT_0 + nCount -1)
             */
            if ((WAIT_OBJECT_0 < event_status) && event_status < ((WAIT_OBJECT_0 + length - 1)))
            {
                /* the event at event_array[event_status - WAIT_OBJECT_0] has been signaled */
                /* It is possible that more than one event was signalled. In that case, (event_status - WAIT_OBJECT_0)
                 * is the index with the lowest event that was signalled. More signalled events can be found higher
                 *
                 * According to the documentation, WAIT_OBJECT_0 is defined as 0
                 */
                if (offset == 0)
                {
                    /* Notification about changes regarding the tun devices.
                     * Or the object is destroyed.
                     * We need to rebuild the array. So exit and rebuild. */
                    /* Cleanup
                     *  Starts with 1 to skip over the dummy
                     */
                    for(j=1;j<i;j++)
                    {
                        /* stop all asynchronous IO */
                        CancelIo(bundle_array[j].fileHandle);
                        CloseHandle(bundle_array[j].overlapped->hEvent);
                        memset(bundle_array[j].buffer.ptr, 0, bundle_array[j].buffer.len);
                        free(bundle_array[j].buffer.ptr);
                        ResetEvent(event_array[j]);
                        CloseHandle(event_array[j]);
                    }
                    /* exit */
                    return JOB_REQUEUE_DIRECT;
                }
                /* The arrays have the same length and the same positioning of the elements.
                 * Therefore, if event_array[j] is signaled, the read on bundle_array[i].fileHandle has succeeded
                 * and bundle_array[j].buffer has our data now.
                 */

                char foo[(bundle_array[offset].buffer.len *4)/3 + 1];
                memset(foo, 0, (bundle_array[offset].buffer.len *4)/3 + 1);
                chunk_to_base64(bundle_array[offset].buffer, foo);

                ip_packet_t *packet;
                /* clone the buffer */
                chunk_t buffer_clone = chunk_clone (bundle_array[offset].buffer);
                packet = ip_packet_create(buffer_clone);
                if (packet)
                {
                        ipsec->processor->queue_outbound(ipsec->processor, packet);
                }
                else
                {
                        DBG2(DBG_ESP, "invalid IP packet read from TUN device");
                }
                /* Reset the overlapped structure, event and buffer */
                /* Print out the package for debugging */
                /* Don't leak packets */
                memset(bundle_array[offset].buffer.ptr, 0, bundle_array[offset].buffer.len);
                memset(bundle_array[offset].overlapped, 0, sizeof(OVERLAPPED));

                if (!start_read(&bundle_array[offset], bundle_array[offset].overlapped->hEvent))
                {
                   /* Cleanup
                    *  Starts with 1 to skip over the dummy
                    */
                    for(j=1;j<i;j++)
                    {
                        /* stop all asynchronous IO */
                        CancelIo(bundle_array[j].fileHandle);
                        CloseHandle(bundle_array[j].overlapped->hEvent);
                        memset(bundle_array[j].buffer.ptr, 0, bundle_array[j].buffer.len);
                        free(bundle_array[j].buffer.ptr);
                    }
                    this->lock->unlock(this->lock);
                    return JOB_REQUEUE_FAIR;
                }
            }
            /* Function failed */
            else
            {
                DBG2(DBG_ESP, "waiting for events on the tun device reads failed.");

                /* Cleanup
                 *  Starts with 1 to skip over the dummy
                 */
                for(j=1;j<i;j++)
                {
                    /* stop all asynchronous IO */
                    CancelIo(bundle_array[j].fileHandle);
                    CloseHandle(bundle_array[j].overlapped->hEvent);
                    memset(bundle_array[j].buffer.ptr, 0, bundle_array[j].buffer.len);
                    free(bundle_array[j].buffer.ptr);
                    ResetEvent(event_array[j]);
                    CloseHandle(event_array[j]);
                }
                this->lock->unlock(this->lock);
                return JOB_REQUEUE_FAIR;

            }
        }
        this->lock->unlock(this->lock);
        return JOB_REQUEUE_DIRECT;
#else
        [...]
#endif /* WIN32 */
}
\end{lstlisting}

\begin{lstlisting}[caption=Debug-Log; Zeigt Problematik mit WaitForSingleObject(),label=lst:debug-log]
00[DMN] Starting IKE service charon-svc (strongSwan 5.4.1dr1, Windows Client 6.2.9200 (SP 0.0)
00[LIB] plugin 'sha3': loaded successfully
00[LIB] plugin 'md4': loaded successfully
00[LIB] plugin 'nonce': loaded successfully
00[LIB] plugin 'x509': loaded successfully
00[LIB] plugin 'pubkey': loaded successfully
00[LIB] plugin 'pkcs1': loaded successfully
00[LIB] plugin 'dnscert': loaded successfully
00[LIB] plugin 'ipseckey': loaded successfully
00[LIB] plugin 'pem': loaded successfully
00[LIB] plugin 'openssl': loaded successfully
00[LIB] plugin 'files': loaded successfully
00[LIB] Error opening registry key: SYSTEM\CurrentControlSet\Control\Class\{4D36E972-E325-11CE-BFC1-08002BE10318}\Properties
00[LIB] TAP-Windows driver version 9.22 available.
00[LIB] created TUN device: {EDA0C976-3256-4B30-8A92-708D2F643E28}
00[LIB] plugin 'kernel-libipsec': loaded successfully
00[LIB] plugin 'kernel-wfp': loaded successfully
00[LIB] plugin 'kernel-iph': loaded successfully
00[LIB] plugin 'socket-win': loaded successfully
00[LIB] plugin 'vici': loaded successfully
00[LIB] plugin 'eap-identity': loaded successfully
00[LIB] plugin 'eap-gtc': loaded successfully
00[LIB] plugin 'eap-dynamic': loaded successfully
00[LIB] plugin 'eap-tls': loaded successfully
00[LIB] plugin 'eap-peap': loaded successfully
00[LIB] feature CUSTOM:kernel-ipsec in plugin 'kernel-wfp' failed to load
00[LIB] feature PUBKEY:DSA in plugin 'pem' has unmet dependency: PUBKEY:DSA
00[LIB] feature CUSTOM:dnscert in plugin 'dnscert' has unmet dependency: RESOLVER
00[LIB] feature CUSTOM:ipseckey in plugin 'ipseckey' has unmet dependency: RESOLVER
00[LIB] feature PRIVKEY:DSA in plugin 'pem' has unmet dependency: PRIVKEY:DSA
00[LIB] feature PRIVKEY:BLISS in plugin 'pem' has unmet dependency: PRIVKEY:BLISS
00[LIB] feature CERT_DECODE:PGP in plugin 'pem' has unmet dependency: CERT_DECODE:PGP
00[LIB] feature CERT_DECODE:OCSP_REQUEST in plugin 'pem' has unmet dependency: CERT_DECODE:OCSP_REQUEST
00[LIB] unloading plugin 'dnscert' without loaded features
00[LIB] unloading plugin 'ipseckey' without loaded features
00[LIB] unloading plugin 'kernel-wfp' without loaded features
00[LIB] loaded plugins: charon-svc sha3 md4 nonce x509 pubkey pkcs1 pem openssl files kernel-libipsec kernel-iph socket-win vici eap-identity eap-gtc eap-dynamic eap-tls eap-peap
00[LIB] unable to load 8 plugin features (7 due to unmet dependencies)
00[JOB] spawning 16 worker threads
00[LIB] created thread 4016
00[LIB] created thread 3032
00[LIB] created thread 3216
00[LIB] created thread 3876
00[LIB] created thread 3132
00[LIB] created thread 1580
00[LIB] created thread 1244
00[LIB] created thread 4212
00[LIB] created thread 3152
00[LIB] created thread 4208
00[LIB] created thread 2560
00[LIB] created thread 2764
00[LIB] created thread 1604
00[LIB] created thread 4220
00[LIB] created thread 1304
00[LIB] created thread 972
08[ESP] entered handle_plain.
08[ESP] Allocated arrays, opened events
08[ESP] put notification event into index 0
08[ESP] Put TUN {EDA0C976-3256-4B30-8A92-708D2F643E28} event in index 1
08[ESP] Allocated buffer.
08[ESP] Allocated file handle.
08[ESP] Allocated overlapped..
08[ESP] Created event
08[ESP] ReadFile() returned 0
08[ESP] Error 997
08[ESP] Read on tun device is pending.
08[ESP] Enumerating tun devices ...
08[ESP] Waiting for events...
08[ESP] Event triggered with event_status 1
08[ESP] offset == 1
08[ESP] position 1 in array
08[ESP] checking if event is signaled.
08[ESP] WaitForSingleObject returned 258
08[ESP] Event is not signaled.
08[ESP] Waiting for events...
[...]
\end{lstlisting}

\begin{lstlisting}[caption=Ausgabe von ipconfig und route -4 print,label=lst:ipconfigroute4]
C:\Users\Noel\bin>ipconfig

Windows-IP-Konfiguration


Ethernet-Adapter THIS IS A TAP DEVICE:

   Verbindungsspezifisches DNS-Suffix:
   Verbindungslokale IPv6-Adresse  . : fe80::596b:bf92:963f:63a3%8
   IPv4-Adresse  . . . . . . . . . . : 172.16.20.2
   Subnetzmaske  . . . . . . . . . . : 255.255.255.255
   Standardgateway . . . . . . . . . :

Ethernet-Adapter Ethernet:

   Verbindungsspezifisches DNS-Suffix: thermicorp.lan
   IPv6-Adresse. . . . . . . . . . . : 2a02:8071:9282:e600:3031:9c6b:6485:3cc9
   Temporäre IPv6-Adresse. . . . . . : 2a02:8071:9282:e600:5181:db51:f4d7:2afa
   Temporäre IPv6-Adresse. . . . . . : 2a02:8071:9282:e600:7154:ac74:30c4:cbee
   Verbindungslokale IPv6-Adresse  . : fe80::3031:9c6b:6485:3cc9%3
   IPv4-Adresse  . . . . . . . . . . : 192.168.178.218
   Subnetzmaske  . . . . . . . . . . : 255.255.255.0
   Standardgateway . . . . . . . . . : fe80::a96:d7ff:fe85:e002%3
                                       192.168.178.1

Tunneladapter isatap.{EDA0C976-3256-4B30-8A92-708D2F643E28}:

   Medienstatus. . . . . . . . . . . : Medium getrennt
   Verbindungsspezifisches DNS-Suffix:

Tunneladapter Teredo Tunneling Pseudo-Interface:

   Medienstatus. . . . . . . . . . . : Medium getrennt
   Verbindungsspezifisches DNS-Suffix:

Tunneladapter isatap.thermicorp.lan:

   Medienstatus. . . . . . . . . . . : Medium getrennt
   Verbindungsspezifisches DNS-Suffix: thermicorp.lan


C:\Users\Noel\bin>route -4 print
===========================================================================
Schnittstellenliste
  8...00 ff ed a0 c9 76 ......TAP-Windows Adapter V9
  3...08 00 27 ef 0b 0e ......Intel(R) PRO/1000 MT-Desktopadapter
  1...........................Software Loopback Interface 1
  4...00 00 00 00 00 00 00 e0 Microsoft-ISATAP-Adapter
  5...00 00 00 00 00 00 00 e0 Teredo Tunneling Pseudo-Interface
  6...00 00 00 00 00 00 00 e0 Microsoft-ISATAP-Adapter #2
===========================================================================

IPv4-Routentabelle
===========================================================================
Aktive Routen:
     Netzwerkziel    Netzwerkmaske          Gateway    Schnittstelle Metrik
          0.0.0.0          0.0.0.0    192.168.178.1  192.168.178.218     10
        127.0.0.0        255.0.0.0   Auf Verbindung         127.0.0.1    306
        127.0.0.1  255.255.255.255   Auf Verbindung         127.0.0.1    306
  127.255.255.255  255.255.255.255   Auf Verbindung         127.0.0.1    306
      172.16.20.2  255.255.255.255   Auf Verbindung       172.16.20.2    276
      172.16.25.2  255.255.255.255  169.254.128.128      172.16.20.2     30
    192.168.178.0    255.255.255.0   Auf Verbindung   192.168.178.218    266
  192.168.178.218  255.255.255.255   Auf Verbindung   192.168.178.218    266
  192.168.178.255  255.255.255.255   Auf Verbindung   192.168.178.218    266
        224.0.0.0        240.0.0.0   Auf Verbindung         127.0.0.1    306
        224.0.0.0        240.0.0.0   Auf Verbindung       172.16.20.2    276
        224.0.0.0        240.0.0.0   Auf Verbindung   192.168.178.218    266
  255.255.255.255  255.255.255.255   Auf Verbindung         127.0.0.1    306
  255.255.255.255  255.255.255.255   Auf Verbindung       172.16.20.2    276
  255.255.255.255  255.255.255.255   Auf Verbindung   192.168.178.218    266
===========================================================================
Ständige Routen:
  Keine

\end{lstlisting}
\subsection{Lizensierung der Arbeit}
Diese \ac{BA} steht unter der GNU General Public License Version 3 (GPLv3)
und darf unter Berücksichtigung der Lizensvereinbarung der GPLv3 verfielfältigt
und verteilt werden. Der Lizenztext ist in~\autoref{subsec:gplv3} oder 
auf der offiziellen Webseite\footnote{\url{https://www.gnu.org/licenses/gpl.html}}
zu finden.

Das Logo der \ac{HSO} steht unter seiner eigenen Lizenz. Es steht nicht unter der GPLv3.
Der gesamte Code steht unter der URL \url{https://github.com/Thermi/Bachelorarbeit} zur Verfügung.

\begin{centering} 

Diese Arbeit wurde mittels \LaTeX{} erstellt
\end{centering}

\subsection{Lizenz}
\label{subsec:gplv3}
\title{GNU GENERAL PUBLIC LICENSE}
\date{Version 3, 29 June 2007}

\maketitle

\begin{center}
{\parindent 0in

Copyright \copyright\  2007 Free Software Foundation, Inc. \texttt{http://fsf.org/}

\bigskip
Everyone is permitted to copy and distribute verbatim copies of this

license document, but changing it is not allowed.}

\end{center}

\renewcommand{\abstractname}{Preamble}
\begin{abstract}
The GNU General Public License is a free, copyleft license for
software and other kinds of works.

The licenses for most software and other practical works are designed
to take away your freedom to share and change the works.  By contrast,
the GNU General Public License is intended to guarantee your freedom to
share and change all versions of a program--to make sure it remains free
software for all its users.  We, the Free Software Foundation, use the
GNU General Public License for most of our software; it applies also to
any other work released this way by its authors.  You can apply it to
your programs, too.

When we speak of free software, we are referring to freedom, not
price.  Our General Public Licenses are designed to make sure that you
have the freedom to distribute copies of free software (and charge for
them if you wish), that you receive source code or can get it if you
want it, that you can change the software or use pieces of it in new
free programs, and that you know you can do these things.

To protect your rights, we need to prevent others from denying you
these rights or asking you to surrender the rights.  Therefore, you have
certain responsibilities if you distribute copies of the software, or if
you modify it: responsibilities to respect the freedom of others.

For example, if you distribute copies of such a program, whether
gratis or for a fee, you must pass on to the recipients the same
freedoms that you received.  You must make sure that they, too, receive
or can get the source code.  And you must show them these terms so they
know their rights.

Developers that use the GNU GPL protect your rights with two steps:
(1) assert copyright on the software, and (2) offer you this License
giving you legal permission to copy, distribute and/or modify it.

For the developers' and authors' protection, the GPL clearly explains
that there is no warranty for this free software.  For both users' and
authors' sake, the GPL requires that modified versions be marked as
changed, so that their problems will not be attributed erroneously to
authors of previous versions.

Some devices are designed to deny users access to install or run
modified versions of the software inside them, although the manufacturer
can do so.  This is fundamentally incompatible with the aim of
protecting users' freedom to change the software.  The systematic
pattern of such abuse occurs in the area of products for individuals to
use, which is precisely where it is most unacceptable.  Therefore, we
have designed this version of the GPL to prohibit the practice for those
products.  If such problems arise substantially in other domains, we
stand ready to extend this provision to those domains in future versions
of the GPL, as needed to protect the freedom of users.

Finally, every program is threatened constantly by software patents.
States should not allow patents to restrict development and use of
software on general-purpose computers, but in those that do, we wish to
avoid the special danger that patents applied to a free program could
make it effectively proprietary.  To prevent this, the GPL assures that
patents cannot be used to render the program non-free.

The precise terms and conditions for copying, distribution and
modification follow.
\end{abstract}

\begin{center}
{\Large \sc Terms and Conditions}
\end{center}


\begin{enumerate}

\addtocounter{enumi}{-1}

\item Definitions.

``This License'' refers to version 3 of the GNU General Public License.

``Copyright'' also means copyright-like laws that apply to other kinds of
works, such as semiconductor masks.

``The Program'' refers to any copyrightable work licensed under this
License.  Each licensee is addressed as ``you''.  ``Licensees'' and
``recipients'' may be individuals or organizations.

To ``modify'' a work means to copy from or adapt all or part of the work
in a fashion requiring copyright permission, other than the making of an
exact copy.  The resulting work is called a ``modified version'' of the
earlier work or a work ``based on'' the earlier work.

A ``covered work'' means either the unmodified Program or a work based
on the Program.

To ``propagate'' a work means to do anything with it that, without
permission, would make you directly or secondarily liable for
infringement under applicable copyright law, except executing it on a
computer or modifying a private copy.  Propagation includes copying,
distribution (with or without modification), making available to the
public, and in some countries other activities as well.

To ``convey'' a work means any kind of propagation that enables other
parties to make or receive copies.  Mere interaction with a user through
a computer network, with no transfer of a copy, is not conveying.

An interactive user interface displays ``Appropriate Legal Notices''
to the extent that it includes a convenient and prominently visible
feature that (1) displays an appropriate copyright notice, and (2)
tells the user that there is no warranty for the work (except to the
extent that warranties are provided), that licensees may convey the
work under this License, and how to view a copy of this License.  If
the interface presents a list of user commands or options, such as a
menu, a prominent item in the list meets this criterion.

\item Source Code.

The ``source code'' for a work means the preferred form of the work
for making modifications to it.  ``Object code'' means any non-source
form of a work.

A ``Standard Interface'' means an interface that either is an official
standard defined by a recognized standards body, or, in the case of
interfaces specified for a particular programming language, one that
is widely used among developers working in that language.

The ``System Libraries'' of an executable work include anything, other
than the work as a whole, that (a) is included in the normal form of
packaging a Major Component, but which is not part of that Major
Component, and (b) serves only to enable use of the work with that
Major Component, or to implement a Standard Interface for which an
implementation is available to the public in source code form.  A
``Major Component'', in this context, means a major essential component
(kernel, window system, and so on) of the specific operating system
(if any) on which the executable work runs, or a compiler used to
produce the work, or an object code interpreter used to run it.

The ``Corresponding Source'' for a work in object code form means all
the source code needed to generate, install, and (for an executable
work) run the object code and to modify the work, including scripts to
control those activities.  However, it does not include the work's
System Libraries, or general-purpose tools or generally available free
programs which are used unmodified in performing those activities but
which are not part of the work.  For example, Corresponding Source
includes interface definition files associated with source files for
the work, and the source code for shared libraries and dynamically
linked subprograms that the work is specifically designed to require,
such as by intimate data communication or control flow between those
subprograms and other parts of the work.

The Corresponding Source need not include anything that users
can regenerate automatically from other parts of the Corresponding
Source.

The Corresponding Source for a work in source code form is that
same work.

\item Basic Permissions.

All rights granted under this License are granted for the term of
copyright on the Program, and are irrevocable provided the stated
conditions are met.  This License explicitly affirms your unlimited
permission to run the unmodified Program.  The output from running a
covered work is covered by this License only if the output, given its
content, constitutes a covered work.  This License acknowledges your
rights of fair use or other equivalent, as provided by copyright law.

You may make, run and propagate covered works that you do not
convey, without conditions so long as your license otherwise remains
in force.  You may convey covered works to others for the sole purpose
of having them make modifications exclusively for you, or provide you
with facilities for running those works, provided that you comply with
the terms of this License in conveying all material for which you do
not control copyright.  Those thus making or running the covered works
for you must do so exclusively on your behalf, under your direction
and control, on terms that prohibit them from making any copies of
your copyrighted material outside their relationship with you.

Conveying under any other circumstances is permitted solely under
the conditions stated below.  Sublicensing is not allowed; section 10
makes it unnecessary.

\item Protecting Users' Legal Rights From Anti-Circumvention Law.

No covered work shall be deemed part of an effective technological
measure under any applicable law fulfilling obligations under article
11 of the WIPO copyright treaty adopted on 20 December 1996, or
similar laws prohibiting or restricting circumvention of such
measures.

When you convey a covered work, you waive any legal power to forbid
circumvention of technological measures to the extent such circumvention
is effected by exercising rights under this License with respect to
the covered work, and you disclaim any intention to limit operation or
modification of the work as a means of enforcing, against the work's
users, your or third parties' legal rights to forbid circumvention of
technological measures.

\item Conveying Verbatim Copies.

You may convey verbatim copies of the Program's source code as you
receive it, in any medium, provided that you conspicuously and
appropriately publish on each copy an appropriate copyright notice;
keep intact all notices stating that this License and any
non-permissive terms added in accord with section 7 apply to the code;
keep intact all notices of the absence of any warranty; and give all
recipients a copy of this License along with the Program.

You may charge any price or no price for each copy that you convey,
and you may offer support or warranty protection for a fee.

\item Conveying Modified Source Versions.

You may convey a work based on the Program, or the modifications to
produce it from the Program, in the form of source code under the
terms of section 4, provided that you also meet all of these conditions:
  \begin{enumerate}
  \item The work must carry prominent notices stating that you modified
  it, and giving a relevant date.

  \item The work must carry prominent notices stating that it is
  released under this License and any conditions added under section
  7.  This requirement modifies the requirement in section 4 to
  ``keep intact all notices''.

  \item You must license the entire work, as a whole, under this
  License to anyone who comes into possession of a copy.  This
  License will therefore apply, along with any applicable section 7
  additional terms, to the whole of the work, and all its parts,
  regardless of how they are packaged.  This License gives no
  permission to license the work in any other way, but it does not
  invalidate such permission if you have separately received it.

  \item If the work has interactive user interfaces, each must display
  Appropriate Legal Notices; however, if the Program has interactive
  interfaces that do not display Appropriate Legal Notices, your
  work need not make them do so.
\end{enumerate}
A compilation of a covered work with other separate and independent
works, which are not by their nature extensions of the covered work,
and which are not combined with it such as to form a larger program,
in or on a volume of a storage or distribution medium, is called an
``aggregate'' if the compilation and its resulting copyright are not
used to limit the access or legal rights of the compilation's users
beyond what the individual works permit.  Inclusion of a covered work
in an aggregate does not cause this License to apply to the other
parts of the aggregate.

\item Conveying Non-Source Forms.

You may convey a covered work in object code form under the terms
of sections 4 and 5, provided that you also convey the
machine-readable Corresponding Source under the terms of this License,
in one of these ways:
  \begin{enumerate}
  \item Convey the object code in, or embodied in, a physical product
  (including a physical distribution medium), accompanied by the
  Corresponding Source fixed on a durable physical medium
  customarily used for software interchange.

  \item Convey the object code in, or embodied in, a physical product
  (including a physical distribution medium), accompanied by a
  written offer, valid for at least three years and valid for as
  long as you offer spare parts or customer support for that product
  model, to give anyone who possesses the object code either (1) a
  copy of the Corresponding Source for all the software in the
  product that is covered by this License, on a durable physical
  medium customarily used for software interchange, for a price no
  more than your reasonable cost of physically performing this
  conveying of source, or (2) access to copy the
  Corresponding Source from a network server at no charge.

  \item Convey individual copies of the object code with a copy of the
  written offer to provide the Corresponding Source.  This
  alternative is allowed only occasionally and noncommercially, and
  only if you received the object code with such an offer, in accord
  with subsection 6b.

  \item Convey the object code by offering access from a designated
  place (gratis or for a charge), and offer equivalent access to the
  Corresponding Source in the same way through the same place at no
  further charge.  You need not require recipients to copy the
  Corresponding Source along with the object code.  If the place to
  copy the object code is a network server, the Corresponding Source
  may be on a different server (operated by you or a third party)
  that supports equivalent copying facilities, provided you maintain
  clear directions next to the object code saying where to find the
  Corresponding Source.  Regardless of what server hosts the
  Corresponding Source, you remain obligated to ensure that it is
  available for as long as needed to satisfy these requirements.

  \item Convey the object code using peer-to-peer transmission, provided
  you inform other peers where the object code and Corresponding
  Source of the work are being offered to the general public at no
  charge under subsection 6d.
  \end{enumerate}

A separable portion of the object code, whose source code is excluded
from the Corresponding Source as a System Library, need not be
included in conveying the object code work.

A ``User Product'' is either (1) a ``consumer product'', which means any
tangible personal property which is normally used for personal, family,
or household purposes, or (2) anything designed or sold for incorporation
into a dwelling.  In determining whether a product is a consumer product,
doubtful cases shall be resolved in favor of coverage.  For a particular
product received by a particular user, ``normally used'' refers to a
typical or common use of that class of product, regardless of the status
of the particular user or of the way in which the particular user
actually uses, or expects or is expected to use, the product.  A product
is a consumer product regardless of whether the product has substantial
commercial, industrial or non-consumer uses, unless such uses represent
the only significant mode of use of the product.

``Installation Information'' for a User Product means any methods,
procedures, authorization keys, or other information required to install
and execute modified versions of a covered work in that User Product from
a modified version of its Corresponding Source.  The information must
suffice to ensure that the continued functioning of the modified object
code is in no case prevented or interfered with solely because
modification has been made.

If you convey an object code work under this section in, or with, or
specifically for use in, a User Product, and the conveying occurs as
part of a transaction in which the right of possession and use of the
User Product is transferred to the recipient in perpetuity or for a
fixed term (regardless of how the transaction is characterized), the
Corresponding Source conveyed under this section must be accompanied
by the Installation Information.  But this requirement does not apply
if neither you nor any third party retains the ability to install
modified object code on the User Product (for example, the work has
been installed in ROM).

The requirement to provide Installation Information does not include a
requirement to continue to provide support service, warranty, or updates
for a work that has been modified or installed by the recipient, or for
the User Product in which it has been modified or installed.  Access to a
network may be denied when the modification itself materially and
adversely affects the operation of the network or violates the rules and
protocols for communication across the network.

Corresponding Source conveyed, and Installation Information provided,
in accord with this section must be in a format that is publicly
documented (and with an implementation available to the public in
source code form), and must require no special password or key for
unpacking, reading or copying.

\item Additional Terms.

``Additional permissions'' are terms that supplement the terms of this
License by making exceptions from one or more of its conditions.
Additional permissions that are applicable to the entire Program shall
be treated as though they were included in this License, to the extent
that they are valid under applicable law.  If additional permissions
apply only to part of the Program, that part may be used separately
under those permissions, but the entire Program remains governed by
this License without regard to the additional permissions.

When you convey a copy of a covered work, you may at your option
remove any additional permissions from that copy, or from any part of
it.  (Additional permissions may be written to require their own
removal in certain cases when you modify the work.)  You may place
additional permissions on material, added by you to a covered work,
for which you have or can give appropriate copyright permission.

Notwithstanding any other provision of this License, for material you
add to a covered work, you may (if authorized by the copyright holders of
that material) supplement the terms of this License with terms:
  \begin{enumerate}
  \item Disclaiming warranty or limiting liability differently from the
  terms of sections 15 and 16 of this License; or

  \item Requiring preservation of specified reasonable legal notices or
  author attributions in that material or in the Appropriate Legal
  Notices displayed by works containing it; or

  \item Prohibiting misrepresentation of the origin of that material, or
  requiring that modified versions of such material be marked in
  reasonable ways as different from the original version; or

  \item Limiting the use for publicity purposes of names of licensors or
  authors of the material; or

  \item Declining to grant rights under trademark law for use of some
  trade names, trademarks, or service marks; or

  \item Requiring indemnification of licensors and authors of that
  material by anyone who conveys the material (or modified versions of
  it) with contractual assumptions of liability to the recipient, for
  any liability that these contractual assumptions directly impose on
  those licensors and authors.
  \end{enumerate}

All other non-permissive additional terms are considered ``further
restrictions'' within the meaning of section 10.  If the Program as you
received it, or any part of it, contains a notice stating that it is
governed by this License along with a term that is a further
restriction, you may remove that term.  If a license document contains
a further restriction but permits relicensing or conveying under this
License, you may add to a covered work material governed by the terms
of that license document, provided that the further restriction does
not survive such relicensing or conveying.

If you add terms to a covered work in accord with this section, you
must place, in the relevant source files, a statement of the
additional terms that apply to those files, or a notice indicating
where to find the applicable terms.

Additional terms, permissive or non-permissive, may be stated in the
form of a separately written license, or stated as exceptions;
the above requirements apply either way.

\item Termination.

You may not propagate or modify a covered work except as expressly
provided under this License.  Any attempt otherwise to propagate or
modify it is void, and will automatically terminate your rights under
this License (including any patent licenses granted under the third
paragraph of section 11).

However, if you cease all violation of this License, then your
license from a particular copyright holder is reinstated (a)
provisionally, unless and until the copyright holder explicitly and
finally terminates your license, and (b) permanently, if the copyright
holder fails to notify you of the violation by some reasonable means
prior to 60 days after the cessation.

Moreover, your license from a particular copyright holder is
reinstated permanently if the copyright holder notifies you of the
violation by some reasonable means, this is the first time you have
received notice of violation of this License (for any work) from that
copyright holder, and you cure the violation prior to 30 days after
your receipt of the notice.

Termination of your rights under this section does not terminate the
licenses of parties who have received copies or rights from you under
this License.  If your rights have been terminated and not permanently
reinstated, you do not qualify to receive new licenses for the same
material under section 10.

\item Acceptance Not Required for Having Copies.

You are not required to accept this License in order to receive or
run a copy of the Program.  Ancillary propagation of a covered work
occurring solely as a consequence of using peer-to-peer transmission
to receive a copy likewise does not require acceptance.  However,
nothing other than this License grants you permission to propagate or
modify any covered work.  These actions infringe copyright if you do
not accept this License.  Therefore, by modifying or propagating a
covered work, you indicate your acceptance of this License to do so.

\item Automatic Licensing of Downstream Recipients.

Each time you convey a covered work, the recipient automatically
receives a license from the original licensors, to run, modify and
propagate that work, subject to this License.  You are not responsible
for enforcing compliance by third parties with this License.

An ``entity transaction'' is a transaction transferring control of an
organization, or substantially all assets of one, or subdividing an
organization, or merging organizations.  If propagation of a covered
work results from an entity transaction, each party to that
transaction who receives a copy of the work also receives whatever
licenses to the work the party's predecessor in interest had or could
give under the previous paragraph, plus a right to possession of the
Corresponding Source of the work from the predecessor in interest, if
the predecessor has it or can get it with reasonable efforts.

You may not impose any further restrictions on the exercise of the
rights granted or affirmed under this License.  For example, you may
not impose a license fee, royalty, or other charge for exercise of
rights granted under this License, and you may not initiate litigation
(including a cross-claim or counterclaim in a lawsuit) alleging that
any patent claim is infringed by making, using, selling, offering for
sale, or importing the Program or any portion of it.

\item Patents.

A ``contributor'' is a copyright holder who authorizes use under this
License of the Program or a work on which the Program is based.  The
work thus licensed is called the contributor's ``contributor version''.

A contributor's ``essential patent claims'' are all patent claims
owned or controlled by the contributor, whether already acquired or
hereafter acquired, that would be infringed by some manner, permitted
by this License, of making, using, or selling its contributor version,
but do not include claims that would be infringed only as a
consequence of further modification of the contributor version.  For
purposes of this definition, ``control'' includes the right to grant
patent sublicenses in a manner consistent with the requirements of
this License.

Each contributor grants you a non-exclusive, worldwide, royalty-free
patent license under the contributor's essential patent claims, to
make, use, sell, offer for sale, import and otherwise run, modify and
propagate the contents of its contributor version.

In the following three paragraphs, a ``patent license'' is any express
agreement or commitment, however denominated, not to enforce a patent
(such as an express permission to practice a patent or covenant not to
sue for patent infringement).  To ``grant'' such a patent license to a
party means to make such an agreement or commitment not to enforce a
patent against the party.

If you convey a covered work, knowingly relying on a patent license,
and the Corresponding Source of the work is not available for anyone
to copy, free of charge and under the terms of this License, through a
publicly available network server or other readily accessible means,
then you must either (1) cause the Corresponding Source to be so
available, or (2) arrange to deprive yourself of the benefit of the
patent license for this particular work, or (3) arrange, in a manner
consistent with the requirements of this License, to extend the patent
license to downstream recipients.  ``Knowingly relying'' means you have
actual knowledge that, but for the patent license, your conveying the
covered work in a country, or your recipient's use of the covered work
in a country, would infringe one or more identifiable patents in that
country that you have reason to believe are valid.

If, pursuant to or in connection with a single transaction or
arrangement, you convey, or propagate by procuring conveyance of, a
covered work, and grant a patent license to some of the parties
receiving the covered work authorizing them to use, propagate, modify
or convey a specific copy of the covered work, then the patent license
you grant is automatically extended to all recipients of the covered
work and works based on it.

A patent license is ``discriminatory'' if it does not include within
the scope of its coverage, prohibits the exercise of, or is
conditioned on the non-exercise of one or more of the rights that are
specifically granted under this License.  You may not convey a covered
work if you are a party to an arrangement with a third party that is
in the business of distributing software, under which you make payment
to the third party based on the extent of your activity of conveying
the work, and under which the third party grants, to any of the
parties who would receive the covered work from you, a discriminatory
patent license (a) in connection with copies of the covered work
conveyed by you (or copies made from those copies), or (b) primarily
for and in connection with specific products or compilations that
contain the covered work, unless you entered into that arrangement,
or that patent license was granted, prior to 28 March 2007.

Nothing in this License shall be construed as excluding or limiting
any implied license or other defenses to infringement that may
otherwise be available to you under applicable patent law.

\item No Surrender of Others' Freedom.

If conditions are imposed on you (whether by court order, agreement or
otherwise) that contradict the conditions of this License, they do not
excuse you from the conditions of this License.  If you cannot convey a
covered work so as to satisfy simultaneously your obligations under this
License and any other pertinent obligations, then as a consequence you may
not convey it at all.  For example, if you agree to terms that obligate you
to collect a royalty for further conveying from those to whom you convey
the Program, the only way you could satisfy both those terms and this
License would be to refrain entirely from conveying the Program.

\item Use with the GNU Affero General Public License.

Notwithstanding any other provision of this License, you have
permission to link or combine any covered work with a work licensed
under version 3 of the GNU Affero General Public License into a single
combined work, and to convey the resulting work.  The terms of this
License will continue to apply to the part which is the covered work,
but the special requirements of the GNU Affero General Public License,
section 13, concerning interaction through a network will apply to the
combination as such.

\item Revised Versions of this License.

The Free Software Foundation may publish revised and/or new versions of
the GNU General Public License from time to time.  Such new versions will
be similar in spirit to the present version, but may differ in detail to
address new problems or concerns.

Each version is given a distinguishing version number.  If the
Program specifies that a certain numbered version of the GNU General
Public License ``or any later version'' applies to it, you have the
option of following the terms and conditions either of that numbered
version or of any later version published by the Free Software
Foundation.  If the Program does not specify a version number of the
GNU General Public License, you may choose any version ever published
by the Free Software Foundation.

If the Program specifies that a proxy can decide which future
versions of the GNU General Public License can be used, that proxy's
public statement of acceptance of a version permanently authorizes you
to choose that version for the Program.

Later license versions may give you additional or different
permissions.  However, no additional obligations are imposed on any
author or copyright holder as a result of your choosing to follow a
later version.

\item Disclaimer of Warranty.

\begin{sloppypar}
 THERE IS NO WARRANTY FOR THE PROGRAM, TO THE EXTENT PERMITTED BY
 APPLICABLE LAW.  EXCEPT WHEN OTHERWISE STATED IN WRITING THE
 COPYRIGHT HOLDERS AND/OR OTHER PARTIES PROVIDE THE PROGRAM ``AS IS''
 WITHOUT WARRANTY OF ANY KIND, EITHER EXPRESSED OR IMPLIED,
 INCLUDING, BUT NOT LIMITED TO, THE IMPLIED WARRANTIES OF
 MERCHANTABILITY AND FITNESS FOR A PARTICULAR PURPOSE.  THE ENTIRE
 RISK AS TO THE QUALITY AND PERFORMANCE OF THE PROGRAM IS WITH YOU.
 SHOULD THE PROGRAM PROVE DEFECTIVE, YOU ASSUME THE COST OF ALL
 NECESSARY SERVICING, REPAIR OR CORRECTION.
\end{sloppypar}

\item Limitation of Liability.

 IN NO EVENT UNLESS REQUIRED BY APPLICABLE LAW OR AGREED TO IN
 WRITING WILL ANY COPYRIGHT HOLDER, OR ANY OTHER PARTY WHO MODIFIES
 AND/OR CONVEYS THE PROGRAM AS PERMITTED ABOVE, BE LIABLE TO YOU FOR
 DAMAGES, INCLUDING ANY GENERAL, SPECIAL, INCIDENTAL OR CONSEQUENTIAL
 DAMAGES ARISING OUT OF THE USE OR INABILITY TO USE THE PROGRAM
 (INCLUDING BUT NOT LIMITED TO LOSS OF DATA OR DATA BEING RENDERED
 INACCURATE OR LOSSES SUSTAINED BY YOU OR THIRD PARTIES OR A FAILURE
 OF THE PROGRAM TO OPERATE WITH ANY OTHER PROGRAMS), EVEN IF SUCH
 HOLDER OR OTHER PARTY HAS BEEN ADVISED OF THE POSSIBILITY OF SUCH
 DAMAGES.

\item Interpretation of Sections 15 and 16.

If the disclaimer of warranty and limitation of liability provided
above cannot be given local legal effect according to their terms,
reviewing courts shall apply local law that most closely approximates
an absolute waiver of all civil liability in connection with the
Program, unless a warranty or assumption of liability accompanies a
copy of the Program in return for a fee.

\begin{center}
{\Large\sc End of Terms and Conditions}

\bigskip
How to Apply These Terms to Your New Programs
\end{center}

If you develop a new program, and you want it to be of the greatest
possible use to the public, the best way to achieve this is to make it
free software which everyone can redistribute and change under these terms.

To do so, attach the following notices to the program.  It is safest
to attach them to the start of each source file to most effectively
state the exclusion of warranty; and each file should have at least
the ``copyright'' line and a pointer to where the full notice is found.

{\footnotesize
\begin{verbatim}
<one line to give the program's name and a brief idea of what it does.>

Copyright (C) <textyear>  <name of author>

This program is free software: you can redistribute it and/or modify
it under the terms of the GNU General Public License as published by
the Free Software Foundation, either version 3 of the License, or
(at your option) any later version.

This program is distributed in the hope that it will be useful,
but WITHOUT ANY WARRANTY; without even the implied warranty of
MERCHANTABILITY or FITNESS FOR A PARTICULAR PURPOSE.  See the
GNU General Public License for more details.

You should have received a copy of the GNU General Public License
along with this program.  If not, see <http://www.gnu.org/licenses/>.
\end{verbatim}
}

Also add information on how to contact you by electronic and paper mail.

If the program does terminal interaction, make it output a short
notice like this when it starts in an interactive mode:

{\footnotesize
\begin{verbatim}
<program>  Copyright (C) <year>  <name of author>

This program comes with ABSOLUTELY NO WARRANTY; for details type `show w'.
This is free software, and you are welcome to redistribute it
under certain conditions; type `show c' for details.
\end{verbatim}
}

The hypothetical commands {\tt show w} and {\tt show c} should show
the appropriate
parts of the General Public License.  Of course, your program's commands
might be different; for a GUI interface, you would use an ``about box''.

You should also get your employer (if you work as a programmer) or
school, if any, to sign a ``copyright disclaimer'' for the program, if
necessary.  For more information on this, and how to apply and follow
the GNU GPL, see \texttt{http://www.gnu.org/licenses/}.

The GNU General Public License does not permit incorporating your
program into proprietary programs.  If your program is a subroutine
library, you may consider it more useful to permit linking proprietary
applications with the library.  If this is what you want to do, use
the GNU Lesser General Public License instead of this License.  But
first, please read \texttt{http://www.gnu.org/philosophy/why-not-lgpl.html}.

\end{enumerate}

