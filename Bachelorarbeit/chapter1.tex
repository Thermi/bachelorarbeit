% This file is part of Bachelorarbeit

% Bachelorarbeit is free software: you can redistribute it and/or modify
% it under the terms of the GNU General Public License version 3 as published by
% the Free Software Foundation.

% Bachelorarbeit is distributed in the hope that it will be useful,
% but WITHOUT ANY WARRANTY; without even the implied warranty of
% MERCHANTABILITY or FITNESS FOR A PARTICULAR PURPOSE.  See the
% GNU General Public License for more details.

% You should have received a copy of the GNU General Public License
% along with Foobar. If not, see <http://www.gnu.org/licenses/>.
\section{Grundlage}
Um IPsec zu verstehen, müssen zuerst die Netzwerkgrundlagen verstanden werden.
\subsection{Netzwerke}
Moderne Computernetzwerke basieren auf Ethernet und \ac{IPv4} oder \ac{IPv6}, wie standardisiert
von der \ac{IETF} in diversen \acp{RFC}.
Jeder Computer im Netzwerk hat mindestens eine sogenannte \ac{NIC}.
Jede \ac{NIC} kann eine oder mehrere physikalische Netzwerkschnittstellen haben,
die den Computer mit dem Netzwerk verbindet. Diese Schnittstelle hat eine einzigartige
Adresse (Die MAC-Adresse), die sie im Netzwerk identifiziert, sowie eine gewisse Bandbreite, mit
der Daten empfangen oder gesendet werden können. Diese Bandbreite hängt von der Schnittstelle, den Kabeltypen und der Kabelqualität ab.
Wenn der Computer unter Benutzung der Schnittstelle Daten an einen anderen Computer im Netzwerk verschickt, wird diese Adresse genutzt, um ihn als
Quelle zu identifizieren und dem anderen Computer zu ermöglichen, den Sender der Daten zu erfahren.
Die MAC-Adresse des Empfängers muss vor dem Senden der Daten erfahren werden, um den Empfänger korrekt spezifizieren zu können.
Die Alternative dazu wäre, die Daten an alle Teilnehmer zu senden, was jedoch ineffizient ist, da
damit die Bandbreite aller anderen Teilnehmer im Netzwerk beansprucht werden würde.
% TODO: Grafik mit Sterntopografie einfügen
In der Realität werden in Computernetzwerken verschiedene Netzwerke für verschiedene Zwecke verwendet.
% TODO: IP. ARP, TCP, UDP
\subsection{Bridging}
Bridging ist die Low-Tech-Lösung, um mehrere Computer zu verbinden ohne mindestens ebensoviele
Kabel und Netzwerkschnittstellen zu benötigen.

%TODO: Grafik mit Bridging einfügen
\subsection{Switching}
%TODO: Grafik mit Switching einfügen
\subsection{Routing}
%TODO: Grafik mit Routing einfügen
\subsection{IPsec}
%Protokollunabhängig
\ac{IPsec} ist ein Konzept zum Absichern von beliebigen \ac{IP}-Paketen oder deren Nutzlasten.
Es stellt Vertraulichkeit, Authentizität und Schutz vor Replay-Attacken bereit.
Die Implementierung der Schutzmechanismen beruht auf sogenannten \acp{SA} und \ac{SP},
die genutzt werden um die Daten zu schützen.
%Kernelspace
%ISAKMP/IKEv1
%Agressive Mode
%Main mode
%PSK-Problem
%IKEv2
%AH
%ESP
%NAT-T
\subsubsection{Route Based}
Route based heißt, dass Pakete, die über das VPN
gesendet werden sollen, in eine virtuelle Netzwerkschnittstelle geroutet werden.
Da gewöhnliches Routing auf Basis der Zieladresse geschieht, ermöglicht eine solche Implementierung
keine Protokol- oder Port-Selektoren.

Implementierungen von IPsec auf Endgeräten sind in der Regel Route based, gleichzeitig werden
die ausgehandelten \acp{SP} jedoch weiterhin erzwungen.
Route based \acp{VPN} werden für gewöhnlich dann genutzt, wenn dynamisches Routen genutzt wird,
wie zum Beispiel \ac{BGP}, \ac{ISIS} oder \ac{OSPF}, da beim Verändern der Routen die \acp{SP} nicht verändert
und damit keine neuen CHILD\_SAs ausgehandelt werden müssen. Ein weiterer Vorteil ist, dass Neulinge es leichter haben
den Paketfluss zu verfolgen.
\subsubsection{Policy Based}
Policy based bedeutet, dass die \ac{IPsec}-\acp{SP}, die ausgehandelt wurden, direkt genutzt werden welche Pakete geschützt werden.
Eine solche Implementierung ermöglicht es, \ac{IPsec} zur Absicherung von sehr genau spezifiziertem
Verkehr zu nutzen, wie zum Beispiel nur \ac{ICMP}-Pakete mit Typ 0, Code 0 (Echo Reply), sowie Typ 8 und Code 0 (Echo Request).
Policy based \ac{IPsec} ermöglicht es \ac{IPsec} direkt in den Stack zu integrieren, ohne notwendigerweise die Routen zu verändern.
\subsubsection{IPsec unter Linux}
Unter Linux gibt es zwei verschiedene \ac{IPsec}-Stacks. 
PF\_KEYv2 (KLIPS) rfc2367
XFRM
\subsubsection{IPsec unter Windows}
