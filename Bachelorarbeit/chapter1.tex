% This file is part of Bachelorarbeit

% Bachelorarbeit is free software: you can redistribute it and/or modify
% it under the terms of the GNU General Public License version 3 as published by
% the Free Software Foundation.

% Bachelorarbeit is distributed in the hope that it will be useful,
% but WITHOUT ANY WARRANTY; without even the implied warranty of
% MERCHANTABILITY or FITNESS FOR A PARTICULAR PURPOSE.  See the
% GNU General Public License for more details.

% You should have received a copy of the GNU General Public License
% along with Foobar. If not, see <http://www.gnu.org/licenses/>.
\section{Grundlage}
Um IPsec zu verstehen, müssen zuerst die Netzwerkgrundlagen verstanden werden.
\subsection{Netzwerke}
Moderne Computernetzwerke basieren auf Ethernet und \ac{IPv4} oder \ac{IPv6}, wie standardisiert
von der \ac{IETF} in diversen \acp{RFC}.
Der Empfang und das Senden von Daten geschieht mit Hilfe von Netzwerkschnittstellen, die die Computer
mit dem Netzwerk verbinden.
Die Adressierung der Computer im Netzwerk geschieht mithilfe der \ac{IP}-Adresse des Empfängers.
Um bidirektionale Kommunikation zu ermöglichen, enthält ein \ac{IP}-Paket als Quelle die
\ac{IP}-Adresse des Senders.
Die Übertragung von \ac{IP}-Paketen zwischen verschiedenen Computern findet durch Nutzung
der vorhandenen Übertragungswege statt, sei es Ethernet über entsprechende Kabel oder
über andere Träger, wie Glasfaser oder Funk.
Die Adressierung der Netzwerkschnittstellen erfolgt über deren MAC-Adressen, was für \ac{IPsec}
jedoch nicht relevant ist, da nur \ac{IP}-Pakete oder die Payload davon, also das Protokoll
auf der Transportschicht, geschützt werden.

\subsection{Routing}
Beim Routen eines Pakets wird nach dem Empfang oder Senden eines \ac{IP}-Pakets durch Nutzung
des sogenannten ''Routing Table'' nachgeschlagen, wohin ein Paket weitergeleitet
werden muss, um den Empfänger zu erreichen.
Hierbei wird nach dem längsten passenden Präfix im Routing Table gesucht, um die
korrekte Route zu finden.
Der \ac{TTL}-Wert des Pakets wird überprüft und dekrementiert. Wenn er 0 erreicht,
wird das Paket verworfen und eine \ac{ICMP}-Fehlermeldung an den Absender verschickt.


\subsubsection{Policy Based Routing}
Policy Based Routing ist eine Sonderform des Routings. Hierbei wird nicht die Zieladresse
zum Finden der Route genutzt, eine Regel.

Linux implementiert \ac{PBR} durch die die Nutzung von Regeln, die Pakete auf Basis von entweder Firewall-Markierungen, der Ziel- oder Quelladresse,
dem Wert des \ac{TOS}-Felds oder der benutzten eingehenden oder ausgehenden Netzwerkschnittstelle in bestimmte
Routing-Tabellen umleitet. 
Die Firewallmarkierung kann ist 32 Bit lang und kann in der Firewall (iptables, nftables, ebtables)
auf eigens gewählte Werte und mit eigens gewählten Regeln gesetzt werden.
Linux erlaubt Anwendungen die Firewallmarkierung direkt im Netzwerksockel zu setzen.


Windows implementiert kein \ac{PBR}.

%TODO: Grafik mit Routing einfügen
\subsection{IPsec}
%Protokollunabhängig
\ac{IPsec} ist ein Konzept zum Absichern von beliebigen \ac{IP}-Paketen oder deren Nutzlasten.
Es stellt Vertraulichkeit, Authentizität und Schutz vor Replay-Attacken bereit.
Die Implementierung der Schutzmechanismen beruht auf sogenannten \acp{SA} und \ac{SP},
die genutzt werden um die Daten zu schützen.

\ac{IPsec} an sich lebt nur im Kernel, jedoch wird für die Aushandlung von Sitzungsschlüsseln,
Algorithmen und dem Erneuern derselben eine Komponente im Userland benötigt.

Im Userland existiert in der Regel ein Systemdienst (Daemon), der eine oder
mehrere Versionen von \ac{IKE} spricht und so ermöglicht \ac{IPsec} \acp{SA}
mit anderen Computern auszuhandeln um \ac{IP}-Pakete zu schützen.
Es steht jedem Systemadministrator jedoch frei die \ac{IPsec} \acp{SA} und \acp{SP}
manuell zu konfigurieren.\footcite[][18]{stephen_kent_rfc_2005}

Der Daemon kommuniziert mit dem Kernel und übergibt ihm die ausgehandelten \ac{IPsec}
\acp{SA} und \acp{SP}, die in der \ac{SAD} und \ac{SPD} verwaltet werden.

Die Kommunikation zwischen den Diensten wird über Authentifizierungsverfahren wie \ac{PSK},
Zertifikatsauthentifizierung, RSA- oder ECDSA-Schlüssel oder \ac{EAP} unter Benutzung 
des Oakley-Protokols\footcite{hilarie_k._orman_rfc_1998} abgesichert, welches mithilfe des \ac{DH}-Schlüsselaustauschprotokolls
geheime Schlüssel zwischen den Teilnehmern aushandelt.\footcite[][50]{charlie_kaufman_rfc_2014}\footcite[][8]{douglas_maughan_rfc_1998}



%Kernelspace
%ISAKMP/IKEv1
%Agressive Mode
%Main mode
%PSK-Problem
%IKEv2
%AH
%ESP
%NAT-T
\subsubsection{Route Based}
Route based heißt, dass Pakete, die über das VPN
gesendet werden sollen, in eine virtuelle Netzwerkschnittstelle geroutet werden.
Da gewöhnliches Routing auf Basis der Zieladresse geschieht, ermöglicht eine solche Implementierung
keine Protokol- oder Port-Selektoren.

Implementierungen von IPsec auf Endgeräten sind in der Regel Route based, gleichzeitig werden
die ausgehandelten \acp{SP} jedoch weiterhin erzwungen.
Route based \acp{VPN} werden für gewöhnlich dann genutzt, wenn dynamisches Routen genutzt wird,
wie zum Beispiel \ac{BGP}, \ac{ISIS} oder \ac{OSPF}, da beim Verändern der Routen die \acp{SP} nicht verändert
und damit keine neuen CHILD\_SAs ausgehandelt werden müssen. Ein weiterer Vorteil ist, dass Neulinge es leichter haben
den Paketfluss zu verfolgen.
\subsubsection{Policy Based}
Policy based bedeutet, dass die \ac{IPsec}-\acp{SP}, die ausgehandelt wurden, direkt genutzt werden welche Pakete geschützt werden.
Eine solche Implementierung ermöglicht es, \ac{IPsec} zur Absicherung von sehr genau spezifiziertem
Verkehr zu nutzen, wie zum Beispiel nur \ac{ICMP}-Pakete mit Typ 0, Code 0 (Echo Reply), sowie Typ 8 und Code 0 (Echo Request).
Policy based \ac{IPsec} ermöglicht es \ac{IPsec} direkt in den Stack zu integrieren, ohne notwendigerweise die Routen zu verändern.
\subsubsection{IPsec unter Linux}
Unter Linux gibt es zwei verschiedene \ac{IPsec}-Stacks. 
PF\_KEYv2 (KLIPS) rfc2367
XFRM
\subsubsection{IPsec unter Windows}
Firewall
Agile VPN
Windows XP
