% This file is part of Bachelorarbeit

% Bachelorarbeit is free software: you can redistribute it and/or modify
% it under the terms of the GNU General Public License version 3 as published by
% the Free Software Foundation.

% Bachelorarbeit is distributed in the hope that it will be useful,
% but WITHOUT ANY WARRANTY; without even the implied warranty of
% MERCHANTABILITY or FITNESS FOR A PARTICULAR PURPOSE.  See the
% GNU General Public License for more details.

% You should have received a copy of the GNU General Public License
% along with Foobar. If not, see <http://www.gnu.org/licenses/>.

\section{Fazit}

In der Arbeit wurden die nötigen Änderungen für strongSwan präsentiert, um
libipsec auf Windows lauffähig zu machen. Dies inkludiert die Implementierung
der Routinen für das Öffnen und Konfigurieren von TAP-Geräten, die vom TAP-Windows-Treiber
bereitgestellt werden. Des weiteren wurde die Installation von IP-Adressen getestet
und erweitert, sowie Code zum Lesen und Schreiben von Paketen auf Handles auf Windows.

Im Verlauf der Arbeit wurden mehrere Probleme mit dem C-Standard, sowie mit
verschiedenen Windowsfunktionen gefunden. Sie wurden in der Arbeit dokumentiert
und um sie herumgearbeitet. Es war etwas überraschend, dass so simple Arbeiten,
wie die Implementierung von Multiplexing mittels Events Probleme bereiten.
\\
Es ist ungewiss, ob die nötigen Features für eine Implementierung eines
vollständigen Roadwarrior-Clients unter Windows noch entwickelt werden.


Die gesamte Arbeit ist gemeinfrei und unterliegt der GPLv3. Somit kann sie,
sowie der Quellcode, frei vertrieben werden, in der Hoffnung dass sie jemandem nutzt.
Der Quellcode der Arbeit, sowie komplette Git-Bäume und diffs vom jeweiligen Root-Commit
der jeweiligen Softwareprojekte sind auf der CD mitenthalten.
