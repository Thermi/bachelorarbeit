% This file is part of Bachelorarbeit

% Bachelorarbeit is free software: you can redistribute it and/or modify
% it under the terms of the GNU General Public License version 3 as published by
% the Free Software Foundation.

% Bachelorarbeit is distributed in the hope that it will be useful,
% but WITHOUT ANY WARRANTY; without even the implied warranty of
% MERCHANTABILITY or FITNESS FOR A PARTICULAR PURPOSE.  See the
% GNU General Public License for more details.

% You should have received a copy of the GNU General Public License
% along with Foobar. If not, see <http://www.gnu.org/licenses/>.

\section{Fazit}

In der Arbeit wurden die nötigen Änderungen für strongSwan präsentiert, um
libipsec auf Windows lauffähig zu machen. Dies inkludiert die Implementierung
der Routinen für das Öffnen und Konfigurieren von TAP-Geräten, die vom TAP-Windows-Treiber
bereitgestellt werden. Des weiteren wurde die Installation von IP-Adressen getestet
und erweitert, sowie Code zum Lesen und Schreiben von Paketen auf Handles auf Windows.

Die Lösung ist unvollständig, da das Empfangen und Senden von Paketen noch nicht funktioniert.
Nichtsdestotrotz wurde mit der \ac{BA} ein Großteil der Basis geschaffen um ''strongSwan''
dazu zu befähigen TAP-Geräte zu nutzen und der TAP-Windows-Treiber in seiner Funktionalität
erweitert.

Die Probleme, die die Implementierung schlussendlich scheitern ließen lassen es zu
die Frage zu stellen, ob Anwendungen, die mit mingw64 kompiliert wurden überhaupt
fehlerfrei auf Handles von Usermodedevice-Treiber zugreifen können und ob
nicht weitere Fehler in mingw64 existieren, die das Implementieren der Kompatibilität
mit TAP-Geräten verhindern.
