% utils.tex

% This file is part of Bachelorarbeit

% Bachelorarbeit is free software: you can redistribute it and/or modify
% it under the terms of the GNU General Public License version 3 as published by
% the Free Software Foundation.

% Bachelorarbeit is distributed in the hope that it will be useful,
% but WITHOUT ANY WARRANTY; without even the implied warranty of
% MERCHANTABILITY or FITNESS FOR A PARTICULAR PURPOSE.  See the
% GNU General Public License for more details.

% You should have received a copy of the GNU General Public License
% along with Foobar. If not, see <http://www.gnu.org/licenses/>.

% for crossing out cells. :(
% from https://tex.stackexchange.com/questions/156162/strike-out-a-table-cell
\pagestyle{empty}% for cropping
\usepackage{colortbl}
\usepackage{pgf}
\usepackage{zref-savepos}

\newcounter{NoTableEntry}
\renewcommand*{\theNoTableEntry}{NTE-\the\value{NoTableEntry}}

\newcommand*{\notableentry}{%
  \multicolumn{1}{@{}c@{}|}{%
    \stepcounter{NoTableEntry}%
    \vadjust pre{\zsavepos{\theNoTableEntry t}}% top
    \vadjust{\zsavepos{\theNoTableEntry b}}% bottom
    \zsavepos{\theNoTableEntry l}% left
    %
    \begin{pgfpicture}%
      \pgfsetlinewidth{.4pt}%
      \pgfsetstrokecolor{red}%
      \edef\llx{0sp}%
      \edef\urx{%
        \the\numexpr
          \zposx{\theNoTableEntry r}%
          -\zposx{\theNoTableEntry l}%
        \relax sp%
      }%
      \edef\lly{%
        \the\numexpr
          \zposy{\theNoTableEntry b}%
          -\zposy{\theNoTableEntry l}%
        \relax sp%
      }%
      \edef\ury{%
        \the\numexpr
          \zposy{\theNoTableEntry t}%
          -\zposy{\theNoTableEntry l}%
        \relax sp%
      }%
      \pgfpathmoveto{\pgfpoint{\llx}{\lly}}%
      \pgfpathlineto{\pgfpoint{\urx}{\ury}}%
      \pgfpathmoveto{\pgfpoint{\llx}{\ury}}%
      \pgfpathlineto{\pgfpoint{\urx}{\lly}}%
      \pgfusepath{stroke}%
      \pgfresetboundingbox
    \end{pgfpicture}%
    %
    \hspace{0pt plus 1filll}%
    \zsavepos{\theNoTableEntry r}% right
  }%
}

% End of the stuff for crossing out tables

% start of code for coloring boxes in bytefield

\newcommand{\colorbitbox}[3]{%
\rlap{\bitbox{#2}{\color{#1}\rule{\width}{\height}}}%
\bitbox{#2}{#3}}
\definecolor{lightcyan}{rgb}{0.84,1,1}
\definecolor{lightgreen}{rgb}{0.64,1,0.71}
\definecolor{lightred}{rgb}{1,0.7,0.71}
\definecolor{lightgray}{gray}{0.8}
% example code for usage:
% \begin{bytefield}[bitheight=\widthof{~Sign~},
% boxformatting={\centering\small}]{32}
% \bitheader[endianness=big]{31,23,0} \\
% \colorbitbox{lightcyan}{1}{\rotatebox{90}{Sign}} &
% \colorbitbox{lightgreen}{8}{Exponent} &
% \colorbitbox{lightred}{23}{Mantissa}
% \end{bytefield}

% end of code for coloring boxes in bytefield


% start of utility code for nice handling of SVG files
% manually typed out because the source PDF's text is obfuscated. What a massive giant cuntosaurus.
% http://tug.ctan.org/tex-archive/info/svg-inkscape/InkscapePDFLaTeX.pdf
% Sadly, the export-latex feature of inkscape is totally broken right now and unusable.

\newcommand{\executeiffilenewer}[3] {%
\ifnum\pdfstrcmp{\pdffilemoddate{#1}}%
{\pdffilemoddate{#2}}>0%
{\immediate\write18{#3}}\fi%
}
\newcommand{\includesvg}[1]{%
\executeiffilenewer{#1.svg}{#1.pdf}%
{inkscape -z -D --file=#1.svg %
--export-pdf=#1.pdf --export-latex}%
\input{#1.pdf_tex}%
}

% end of utility code for nice handling of SVG files

% start of the code for nice labels above bitboxes in bytefield

\newcommand{\bitlabel}[2]{%
    \bitbox[]{#1}{%
        \raisebox{0pt}[4ex][0pt]{%
            \turnbox{45}{\fontsize{7}{7}\selectfont#2}%
        }%
    }%
}

% end of the code for nice labels above bitboxes in bytefield
