\documentclass[a4paper]{article}
\usepackage{listing}
\usepackage{amsfonts}
\usepackage{amssymb}
\usepackage{listings}
\usepackage{tabularx}
\usepackage[utf8]{inputenc}
\usepackage[ngerman]{babel}
\usepackage{amsmath}
\usepackage{graphicx}
\usepackage[hidelinks]{hyperref}
\usepackage{pdfpages}
\usepackage{pgfplots}
\usepackage[printonlyused]{acronym}
\usepackage{setspace}

\lstset{
breaklines=true,
literate={ö}{{\"o}}1
         {ä}{{\"a}}1
         {ü}{{\"u}}1
         {ß}{{\ss}}2
         {Ä}{{\"A}}1
         {Ü}{{\"U}}1
         {Ö}{{\"O}}1
}


\setstretch{1.3}
\newcommand*\BlankPage{\newpage\null\thispagestyle{empty}\newpage}


\newcommand{\comment}[1]{}  %comment not shown

\begin{document}

\pagenumbering{roman}


\title{Motivation und Ziele der Bachelorarbeit}
\author{Noel Kuntze}
\clearpage\maketitle
\thispagestyle{empty}

\BlankPage

\phantomsection

\section*{Abkürzungsverzeichnis}
\addcontentsline{toc}{section}{Abkürzungsverzeichnis}
\begin{acronym}
    \acro{ACK}{Acknownledge}
    \acro{ADC}{Analog-Digital Converter}
    \acro{AH}{Authentication Header}
    \acro{AES}{Advanced Encryption Standard}
    
	\acro{API}{Application Programming Interface}
	
	\acro{BA}{Bachelorarbeit}
    \acro{CBC-MAC}{Cipher Block Chaining MAC}
    \acro{CI}{Codeimage}
    \acrodefplural{CI}[CIs]{Codeimages}

    \acro{CPU}{Central Processing Unit}
    \acrodefplural{CPU}[CPUs]{Central Processing Units}
    
	\acro{DH}{Diffie-Hellman}
	
    \acro{DRNG}{Deterministic Random Number Generator}
    \acrodefplural{DRNG}[DRNGs]{Deterministic Random Number Generators}

	\acro{ESP}{Encapsulated Security Payload}
	
	\acro{GUI}{Graphical User Interface}
	
    \acro{HMAC}{Hashed Message Authentication Code}

    \acro{IDE}{Integrated Development Environment}
    
    \acro{IoT}{Internet of Things}
    \acro{IRC}{Internet Relay Chat}
    
    \acro{IV}{Initialisierungsvektor}

    \acro{KB}{Kilo Byte}

    \acro{LAT}{Line Address Table}


    \acro{MAC}{Message Authentication Code}
    \acrodefplural{MAC}[MACs]{Message Authentication Codes}
    \acro{mC}[$\mu$$C$]{Microcontroller}
    \acro{MOO}{Mode of Operation}

    \acro{NACK}{Negative Acknowledge}
    \acro{NED}{Network Description}
    \acro{NDRNG}{Non-deterministic Random Number Generator}
    \acrodefplural{NDRNG}[NDRNGs]{Non-deterministic Random Number Generators}

    \acro{OFB}{Output Feedback Mode}
    \acro{OMNeTpp}[OMNeT$++$]{Objective Modular Network Testbed in C++}

    \acro{OTP}{One Time Pad}
    \acrodefplural{OTP}[OTPs]{One Time Pads}
    
    \acro{PC}{Personal Computer}
    \acro{PRW}{Pseudo Random Words}
    \acro{PRNG}{Pseudo Random Number Generator}
    \acrodefplural{PRNG}[PRNGs]{Pseudo Random Number Generators}

    \acro{RAM}{Random Access Memory}
    \acro{RISC}{Reduced Instruction Set Computing}
    
   	\acro{PFS}{Perfect Forward Secrecy}
    \acro{RNG}{Random Number Generator}
    \acrodefplural{RNG}[RNGs]{Random Number Generators}

    \acro{ROM}{Read Only Memory}

    \acro{ROP}{Return Oriented Programming}

    \acro{SCUBA}{Secure Code Update By Attestation}
    \acro{SHA}{Secure Hashing Algorithm}
    
    \acro{SOC}{System On Chip}
    
    \acro{SWATT}{SoftWare-based ATTestation}

    \acro{TI}{Texas Instruments}
    \acro{TPM}{Trusted Platform Module}

    \acro{TRNG}{True Random Number Generator}
    \acrodefplural{TRNG}[TRNGs]{True Random Number Generators}

	\acro{WFP}{Windows Filtering Platform}
	
	\acro{VICI}{Versatile IKE Configuration Interface}
	
    \acro{WSN}{Wireless Sensor Network}
    \acrodefplural{WSN}[WSNs]{Wireless Sensor Networks}
    \acro{WxorX}[W$\oplus$X]{Write xor Execute}

    \acro{XOR}{Exclusive Or}
\end{acronym}

\newpage

\thispagestyle{empty}
\section{Motivation}
Die Motivation hinter der Bachelorarbeit ist der Mangel eines für Endanwender 
nutzbaren, quelloffenen und vielseitigen IPsec-Clients auf der Windows-Plattform.
Mit dem ''Agile VPN''-Client von Microsoft existiert seit Windows 7 ein nativer
IKEv2-Client, der jedoch nur wenige Authentifizierungsmöglichkeiten unterstützt.

Des weiteren unterstütz er nur die \ac{DH} Schlüsselaushandlung für
Schlüssellänge 1024 Bit, was laut NIST seit 2011 nur unzureichende Sicherheit bietet.
Laut Adrian et al\footnote{\textit{Imperfect Forward Secrecy: How Diffie-Hellman Fails in Practice} \url{https://weakdh.org/imperfect-forward-secrecy-ccs15.pdf}}
kann \c{DH} mit Schlüssellänge 1024 Bit bereits von Nationalstaaten gebrochen werden,
weshalb es in der Praxis keinen ausreichenden Schutz mehr bietet. 
Des weiteren unterstützt der ''Agile VPN''-Client kein \ac{PFS}.

Der Client hat auch verschiedene Funktionale Probleme, so unterstützt er zum Beispiel
keine IKE-Fragmentierung\footnote{Fragmentiert IKE-Pakete nicht auf Netzwerkebene, sondern auf Anwendungsebene},
was zur IKE-Aushandlung mit großen Datenpaketen über defekte\footnote{Netzwerke, in denen der Administrator die kontraproduktive Idee hatte ICMP-Pakete zu verwerfen}
Netzwerke nötig ist und implementiert Traffic-Selector-Narrowing nicht korrekt.

Seit Windows 10 versendet Windows DNS-Anfragen für Domänen an alle konfigurierten
DNS-Resolver, nicht nur an die Resolver, die für das genutzte Interface konfiguriert wurden.
Dies ist ein Problem für die Privatsphäre und kann mit einem \ac{WFP}-Modul
gelöst werden, welches DNS-Pakete verwirft, welche nicht über den VPN-Adapter
geschickt werden.


Das ''strongSwan''-Projekt der HSR Rapperswill stellt eine quelloffene
IKE-Implementierung bereit, welche auf allen größeren Plattformen lauffähig ist
(Microsoft Windows, FreeBSD, Linux, Google Android und Apple Mac OS X). 
Darüber hinaus wurde im Rahmen des Projekts eine Userspace-Implementierung von IPsec entwickelt,
welche auf Linux lauffähig ist. Für die Kontrolle des IPsec-Diensts von strongSwan
existiert seit Version 5.2.0 eine synchrone \ac{API} (\ac{VICI}), die in C implementiert ist,
jedoch mit bereitgestellten Perl, Ruby und Python-Modulen genutzt werden kann.

Auf Basis dieser Implementierung ist es relativ leicht möglich einen ausreichend
sicheren und vielseitigen IPsec-VPN-Client auf Windows-Basis zu entwickeln.
Mit libipsec steht eine Basis für das Verarbeiten von IPsec-Paketen bereit, mit strongSwan eine
bewähre IKE-Implementierung und mit \ac{VICI} eine nutzbare \ac{API} für die Konfiguration.

Eine Portierung auf Windowsw würde in ihren Fähigkeiten durch den Windows-Kernel
beschränkt. So beherrscht er kein regelbasiertes Routen, weshalb das Implementieren
oder Erlauben von Protokollselektoren keinen Sinn ergibt. Eine weitere
Beschränkung ist, dass der Verkehr zum Responder nicht getunnelt werden darf, da
das sonst eine Schleife ergeben würde.

libipsec krankt im Moment noch am Problem, dass das Plugin Pakete im Speicher unbegrenzt
puffert, was dazu führen kann, dass dem Computer der Speicher ausgeht. Für dieses
Problem stehen bereits Patches bereit, die jedoch noch getestet werden müssten.

Die Motivation für das \ac{GUI} Python zu nutzen ist simpel: Ich beherrsche bereits
Python und es lässt sich mit wxPython ein plattformübergreifend nutzbares User Interface erstellen.
Mit dem \ac{VICI} Python Egg lässt sich ''strongSwan'' steuern.

Ich selbst nutze ''strongSwan'' seit mehreren Jahren und helfe im \ac{IRC}-Kanal\footnote{irc://chat.frenode.net/strongswan}
aus. Aus diesem Grund bin ich mit der Materie vertraut und erkenne den Mangel an einem
nutzbaren IPsec-Client auf Windows.

\section{Ziele}
Das Ziel der \ac{BA} ist die Erstellung eines \ac{GUI} für die Konfiguration
und Kontrolle von strongSwan auf Basis von wxPython und \ac{VICI}, sowie
die Anpassung des strongSwan-Codes um als Roadwarrior auf WIndows-Basis nutzbar zu sein.
Darunter fallen etwaige Anpassungen des libipsec-Codes oder die Anpassungen der kernel-wfp
und kernel-iph-Plugins, sowie etwaige Änderungen an anderen Teilen von strongSwan, soweit
benötigt.

\paragraph{Ziele}
Für die Portierung von libipsec:
\begin{itemize}
\item Transformierung von Paketen zu in UDP eingepackten ESP-Paketen
\item Verwalten der Routen zu den getunnelten Netzen
\item Unterstützung des Tunnel-Modus
\item Kein Untersützung von protokolabhängigen Policies
\item Keine Unterstützung für das Schützen von Verkehr direkt zum Server
\end{itemize}

Für das \ac{GUI}:
\begin{itemize}
\item Initierung und Terminierung von IPsec-Tunneln
\item Anzeigen der Tunnelstatus
\item Konfiguration von IKE-Konfigurationen mit IKEv1/IKEv2
\item Konfiguration von IPsec-Tunneln mit den Standardisierten und verfügbaren Kryptographischen Algorithmen
\item Konfiguration von IPsec-Tunneln 
\item Unterstützung von Authentifizierung über RSA-Schlüssel, X.509-Zertifikaten, 
EAP-MSCHAPv2, EAP-GTC, EAP-TLS, EAP-TTLS und Xauth
\item Unterstützung des Tunnel-Modus
\item Anzeigen von Debug-Informationen
\item Neustarten des Diensts
\item Unterstützung für das Laden von Zertifikaten und Textgeheimnissen
\end{itemize}

\paragraph{Auswirkungen}
Durch die Bereitstellung eines flexiblen, quelloffenen IPsec-Clients für Windows
wird die Nutzung von starker Authentifizierung auf Windows ermöglicht, worunter
starke \ac{DH} und ECDH Austausche, sowie die Nutzung von Benutzernamen mit Passwort und \acp{OTP}
fallen.
Es ist zu erwarten, dass es Interesse am Ergebnis der \ac{BA} geben wird, da
Anwender an der Nutzung von sicherer Authentifizierung interessiert sind
und nur an der Nicht-Verfügbarkeit einer entsprechenden Software bei der Nutzung 
scheitern.
\end{document}
